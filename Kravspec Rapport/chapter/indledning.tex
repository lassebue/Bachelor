\thispagestyle{fancy}
\chapter{Indledning}
\label{chp:indledning}

Målet med projektet er at undersøge mulighederne for at lave EMG-signal genkendelse på armens muskler vha. machine learning.
I forbindelse med machine learning, har man behov for relativt store mængder data. Har man ingen data, kan man ikke lave machine learning. I projektet er der behov for et device, der både er i stand til at opfange EMG-signaler fra armen,  og derefter sende brugbare data til en computer e.l.

\section{Systembeskrivelse}
I dette projekt vil en Myo levere denne funktionalitet. En Myo, er et relative billigt wearable device, der har en række forskellige funktionaliteter, hvor en funktionalitet netop er at opfange EMG signaler og sende den opsamlede data til et program. 

Da der er behov for data specifikt til machine learning, vil der i projektet blive designet et software program, med henblik på at opsamle og ordne data sendt fra Myo'en, til senere træning af machine learning modeller.

\subsection{Anvendelse}
En applikation vil blive udviklet der kan vise hvordan systemet kan bruges. Dette er an CrustCrawler applikation. Sammen med en CrustCrawler-robot Vil det blive vist at ved anvendelse af EMG og machine learning, kan CrustCrawlerens klo åbnes og lukkes. Kloen vil kunne symbolisere en protese eller exo-skelet, som ville være et oplagt mål, med 
\subsection{Inspiration}
Projektet er netop inspireret af disse elektriske proteser som anvender EMG signaler til at styring.

\section{Projektafgrænsning}
\label{chp:projektafgensning}
I dette projekt vil der blive fokuseret på tre overordnet dele. Nemlig dataopsamlingen, udvikling af en machine learning model til genkendelse af data sendt fra Myo’en, og et styringssystem til CrustCrawler-robotten, som anvender realtidsgenkendelse af data sendt fra Myo’en til at aktivere robottens gribe og slippe funktioner. 

\myFigure{konceptbillede}{Systemskitse}{fig:koncept}{1}

På systemskitsen i figur \ref{fig:koncept} ses, hvordan systemet overordnet set er opdelt vha. af stiplede linier. Første del af systemet, er dataopsamling, som sker vha. Myo’en gennem et PC program, der bliver udviklet til at gemme og ordne data relateret til de forskellige poses. Programmet gemmer dataene online, således at der er fleksibel tilgang til dem.

I systemets anden del - Model træningen, som ses midt i figuren. Her behandles behandles dataet, så de kan anvendes til træningen. Her trænes en model, som skal kunne genkende poses lavet af brugeren med en Myo. 

Til højre på figur \ref{fig:koncept} ses systemets sidste del, som er selve anvendelse af ML modellen. Her modtages ny data fra en Myo på en computer med en CrustCrawler applikation. Dataene behandles først på samme vis, som ved model træningen, og efterfølgende genkendes af ML modellen. Hvis dataene fra Myo’en genkendes, som poses vil kommandoer blive sendt fra computeren til CrustCrawleren, hvorefter den udfører den pågældende handling.  

Mht. datopsamlingen er det tanken, at dataen skal ordnes efter, hvem og hvilken hånd dataen kommer fra, hvilken orienteringen Myo’en har på testpersonens arm, samt hvilke poses, der er lavet, når dataen opsamles. Videre skal der i projektet opbygges en online database, hvor forskellige brugere af dataopsamlingsprogrammet vil kunne uploade deres data. Således vil andre brugere kunne anvende en større datamængde til træne deres egne machine learnings modeller.

Til træning af machine learnings modellen, er det intentionen at anvende Matlabs indbyggede toolboxes, der er særligt egnet til at udvikle og træne genkendelses modeller, samt at teste disse. 
Her er det ligeledes muligt, at undersøge, hvilke typer af machine learning modeller, der er de meste effektive til at genkende poses, og hvilke poses, de forskellige modeller har sværest ved er se forskel på. 

Fra Matlab vil det muligvis være muligt, at modtage data i realtid, lave databehandling og kør classifications model på den, og dermed på realtid kunne genkende, hvilke poses, der laves med hånden.
