\thispagestyle{fancy}
\chapter{Indledning}
\label{chp:indledning}

\section{Systembeskrivelse}
\subsubsection{Anvendelse}
Systemet giver plads til mange forskellige anvendelsesmuligheder.\\

På Regionshospitalet Hammel Neurocenter Forskes, udvikles og eksperimenteres der en masse inden for nye løsninger til at hjælpe til med genoptræning af patienter. Én af de projekter der har været igang er udviklingen af et en exoskelet-arm. Denne exo-arm skal hjælpe patienten med at bøje albueleddet når bicept bliver aktiveret. Modsat gælder det når det registreres at tricept aktiveres, her hjæper exoarmen med at strække patientens arm ud igen. Exoarmen har ikke med hensigt at gøre patienten stærkere, men har i stedet fokus på genoptræning af patienterne, ved at give dem evnen tilbage til at løfte deres arm. PÅ samme måde som denne hjælper til i albueleddet, vil anvendelsen af vores system være til bevægelse af hånden, og fingrende. Ved at der registreres aktivitet i musklerne i underarmen, vil en exo skelet-hånd kunne hjælpe en patient med at åbne og lukke sin hånd. Derved vil patienten kunne få anvendelsen af sin hånd tilbage, og hende med at bruge den igen, til at grube fat i ting og løfte dem. Det kan siges at systemet her er en udvidelse af den exo-arm der er udviklet i projektet på Neurocentret.

En anden anvendelse, der vil være lidt mere primitiv, men også viser hvad systemet er i stand til, er at bruge det sammen med CrustCrawler-robotten som findes Ingeniørhøjskolens robotlab. Denne robot er en arm der kan dreje og har flere bøjelige led, og i enden en "klo" der kan gribe ud efter noget. Dette kunne være når der registreres en knyttet næve.