\thispagestyle{fancy}
\chapter{Ikke funktionelle krav}
I følgende kapitel beskrives systemets ikke funktionelle krav. Disse har ikke en direkte effekt på systemets funktionalitet, men i højere grad omfanget og kvaliteten af systemet, samt nogle generelle valg, der er truffet i forbindelse med projektet.

\section{Dataopsamling}
Dataopsamlingsprogrammet der sammen med et Myo armband anvendes af testpersonen, som er beskrevet i tabel \ref{tab:testperson}, har følgende ikke funktionelle krav.

\subsection{Brugergrænseflade}
Dataopsamling skal anvendes af testpersonen der har en Myo til opsamling af data. Brugergrænsefladen skal være så enkelt som muligt. \todoall{Ændre her hvis ikke vi gør det ud fra Jakob Nielsens designprincipper} Derfor arbejdes der ud fra Jakob Nielsens design principper. Hvilket gør at testpersonen får en god oplevelse med systemet.

\subsection{Datasamling}
Det ønskes at alle ejerer af et Myo armbånd skal kunne bidrage til systemets datasamling, derfor er det besluttet, at datasamlingen skal være online. Ligeledes vil det, med en online datasamling, kunne lade sig gøre at tilgå dataene fra flere computere samtidigt, således at behov udenfor projektafgrænsninger vil kunne dækkes. På trods af at datasamlingen ligger online, kan der stadig være behov for at opsamle data selvom man ikke er online, og dette skal også kunne gøres, hvorefter det kan uploades når man har internetadgang.

\section{Model træning}
Til systemet bruges Lasse Bue Svendsen og Kristoffer Sloth Gade som tespersoner, så det data vi anvender er målt på Lasse og Kristoffer. Gennem projektet og alle målingerne bruges der altid højre hånd for at gøre det mere enkelt at arbejde med.\\
Når en pose laves skal der gå kortest mulig tid til at den er genkendt af algoritmen. Til anvendelse arbejdes der mod at der bruges én Myo ad gangen.


\section{CrustCrawler}
Af ikke funktoinelle krav til CrustCrawleren er at når åben hånd registreres begynder kloen at åbne, og slappes der af i hånden stoppper kloen der hvor den er kommet til. Kloen fungerer altså ikke som enten åben eller lukket, med at den åbner gradvist når bruger har åben hånd. Bruger vil altså kunne stoppe kloen halves hvis der er brug for dette.