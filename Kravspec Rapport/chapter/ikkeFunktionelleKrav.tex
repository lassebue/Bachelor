%\thispagestyle{fancy}
\chapter{Ikke funktionelle krav}
I følgende kapitel beskrives systemets ikke funktionelle krav. Disse har ikke en direkte effekt på systemets funktionalitet, men i højere grad omfanget og kvaliteten af systemet, samt nogle generelle valg, der er truffet i forbindelse med projektet.

\section{Dataopsamling}
Dataopsamlingsprogrammet der sammen med et Myo armband anvendes af testpersonen, som er beskrevet i tabel \ref{tab:testperson}, har følgende ikke funktionelle krav.

\subsection{Brugergrænseflade}
Dataopsamlings applikationen skal anvendes af testpersonen der har en Myo til opsamling af data. Brugergrænsefladen skal være så enkel som muligt, således at den er let at bruge.

\subsection{Datasamling}
Det ønskes at alle ejerer af et Myo armbånd skal kunne bidrage til systemets datasamling, derfor er det besluttet, at datasamlingen skal være online. Ligeledes vil det, med en online datasamling, kunne lade sig gøre at tilgå dataene fra flere computere samtidigt, således at behov udenfor projektafgrænsninger vil kunne dækkes. På trods af at datasamlingen ligger online, kan der stadig være behov for at opsamle data selvom man ikke er online, og dette skal også kunne gøres i en .csv fil

\section{Model træning}
Til systemet bruges Lasse Bue Svendsen og Kristoffer Sloth Gade som tespersoner, så det data vi anvender er målt på Lasse og Kristoffer. Gennem projektet og alle målingerne bruges der altid højre hånd for at gøre det mere enkelt at arbejde med.\\
Når en pose laves skal der gå kortest mulig tid til at den er genkendt af algoritmen. Til anvendelse arbejdes der mod at der bruges én Myo ad gangen.


\section{Testapplikation}
Testapplikationen skal anvendes af brugeren der har en Myo og CrustCrawler. Brugergrænsefladen skal være så enkel som muligt, således at den er let at bruge.