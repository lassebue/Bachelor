\thispagestyle{fancy}
\chapter{Ikke funktionelle krav}
I følgende kapitel beskrives systemets ikke funktionelle krav. Disse har ikke en direkte effekt på systemets funktionalitet, men i højere grad omfanget og kvaliteten af systemet, samt nogle generelle valg, der er truffet i forbindelse med projektet.

\section{Dataopsamling}
Dataopsamlingsprogrammet der sammen med et Myo armband der anvendes af testpersonen, som er beskrevet i tabel \ref{tab:testperson}, har følgende ikke funktionelle krav.

\subsection{Brugergrænseflade}
Dataopsamling skal anvendes af testpersonen der har en Myo til opsamling af data. Brugergrænsefladen skal være så enkelt som muligt. \todoall{Ændre her hvis ikke vi gør det ud fra Jakob Nielsens designprincipper} Derfor arbejdes der ud fra Jakob Nielsens design principper. Hvilket gør at testpersonen får en god oplevelse med systemet.

\subsection{Datasamling}
Det ønskes at alle ejerer af et Myo armbånd skal kunne bidrage til systemets datasamling, derfor er det besluttet, at datasamlingen skal være online. Ligeledes vil det, med en online datasamling, kunne lade sig gøre at tilgå dataene fra flere computere samtidigt, således at behov udenfor projektafgrænsninger vil kunne dækkes. På trods af at datasamlingen ligger online, kan der stadig være behov for at opsamle data selvom man ikke er online, og dette skal også kunne gøres, hvorefter det kan uploades når man har internetadgang.

\section{Model træning}

\subsection{Systemydelse}
Hvor mange personer, skal vi have data fra? 2
Hvilken hånd højre / venstre? Højre
Hvor hurtig genkendes en pose ? 0.5 s
Hvor mange Myo'er kan håndteres samtidig?



\section{CrustCrawler}

Hvad skal man kunne lave med CrustCrawleren?
Skal man kunne styre en på samme tid?