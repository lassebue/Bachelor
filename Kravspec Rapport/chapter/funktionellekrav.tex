\chapter{Funktionelle Krav}
\label{chp:funktionellen}
I det følgende kapitel vil de funktionelle krave blive fremsat ud fra systembeskrivelsen i sektion \ref{chp:projektafgensning} og de nævnte aktører i sektion \ref{chp:aktorbeskrivelser}. Systemets krav bliver opsat i use cases, som medvirker til at danne overblik over, hvordan systemet krav bliver opfyldt og hvordan de forskellige aktører relaterer til systemet systemet.    
\section{Use Cases}
Diagram

\subsection{Se vejledning}
\bgroup
\def\arraystretch{1.8}
\begin{center}
	\begin{table}[htbp]
		\rowcolors{2}{white}{lightgrey}
		\begin{tabular}{lp{10cm}}
			\rowcolor{grey} Use case 1:	& Se vejledning \\
			Mål:	& At Testpersonen ved hvordan data skal indsamles korrekt. \\
			initiation:	& Testperson\\
			Aktør: & Testperson\\
			Prækondition: & Dataopsamlings applikationen er klar.\\
			Postkondition: & Vejledningen til dataopsamlingen er blevet vist.\\
			Hovedscenarie: & 1: Testperson vælger at se vejledningen.\\
			& 2: Vejledningen vistes i applikationen.\\
		\end{tabular}
		\caption{Use Case 1}
	\end{table}
\end{center}

\subsection{Indsaml datasæt}

\begin{center}
	\begin{table}[htbp]
		\rowcolors{2}{white}{lightgrey}
		\begin{tabular}{lp{10cm}}
			\rowcolor{grey} Use case 2:		& Indsaml datasæt \\
			Mål: 	& At få insamlet data om en pose fra testperson. \\
			initiering:	& Testperson\\
			Aktør: & Testperson\\
			Prækondition: & Testperson har Myo båndet på, og dataindsamlingssoftware kører\\
			Postkondition: & Datasæt er indsamlet\\
			Hovedscenarie: & 1: Testperson trykker på knap for at starte dataindsamlingen\\
			& 2: Testperson laver pose\\
			& 3: Testperson trykker på knap for at stoppe dataindsamlingen\\
			& 4: Admin vælger, hvilken bevægelsestype der er blevet indsamlet og angiver hvem testpersonen er.\\
			& 5: Admin trykke på knap for at gemme datasæt\\
		\end{tabular}
		\caption{Use Case 2}
	\end{table}
\end{center}

\subsection{Tilføj ny bevægelsestype}
\begin{center}
	\begin{table}[htbp]
		\rowcolors{2}{white}{lightgrey}
		\begin{tabular}{lp{10cm}}
			\rowcolor{grey} Use case 3:		& Tilføj ny pose\\
			Mål: 	& At tilføje en ny pose som systemet kan indsamle data om og genkende \\
			initiation:	& Testperson\\
			Aktør: & Testperson\\
			Prækondition: & Datainsamlingssoftware kører \\
			Postkondition: & En ny pose er tilføjet dataindsamlingssoftwaren\\
			Hovedscenarie: & 1:Testperson trykker på knap for at tilføje en ny pose\\
			& 2. Testperson angiver titel for den nye pose\\
			& 3. Testperson vælger, hvilken pose der er blevet indsamlet og angiver hvem testpersonen er\\
			& 4. Testperson trykker på knap for at gemme pose\\
		\end{tabular}
		\caption{Use Case 3}
	\end{table}
\end{center}

\subsection{Slet datasæt}
\begin{center}
	\begin{table}[htbp]
		\rowcolors{2}{white}{lightgrey}
		\begin{tabular}{lp{10cm}}
			\rowcolor{grey} Use case 4:	& Slet datasæt \\
			Mål:	& At slette et gemt datasæt fra databasen \\
			initiation:	& Testperson\\
			Aktør: & Testperson\\
			Prækondition: & Et datasæt eksisterer i databasen\\
			Postkondition: & datasæt er slettet fra databasen\\
			Hovedscenarie: & 1: Testperson vælger eksisterende datasæt\\
			& 2: Testperson trykker på knap for at slette valgte datasæt.
		\end{tabular}
		\caption{Use Case 4}
	\end{table}
\end{center}

\subsection{Anvendelse af pose genkendelse}
\begin{center}
	\begin{table}[htbp]
		\rowcolors{2}{white}{lightgrey}
		\begin{tabular}{lp{10cm}}
			\rowcolor{grey} Use case 5:		& genkendelse af pose\\
			Mål: & At kunne finde ud af hvilken pose, der laves af Brugeren med Myo båndet på. \\
			initiation:	& Bruger \\
			Aktør: & Bruger\\
			Prækondition: & Bruger har Myo bånd på, som er forbundet til pc med genkendelsessoftware kørende \\
			Postkondition: & Pose er blevet genkendt \\
			Hovedscenarie: & 1: Bruger laver håndbevægelse \\
			& 2: Myo-data sendes til software på computeren\\
			& 3: Pose genkendelses software processerer Myo-data\\
			& 4: Myo-data genkendes\\
			& 5: Brugerens pose genkendes\\
		\end{tabular}
		\caption{Use Case 5}
	\end{table}
\end{center}

\subsection{Lukke CrustCrawler greb}
\begin{center}
	\begin{table}[htbp]
		\rowcolors{2}{white}{lightgrey}
		\begin{tabular}{lp{10cm}}
			\rowcolor{grey} Use case 6:		& Lukke CrustCrawler greb \\
			Mål: & Som Bruger vil man lukke CrustCrawlerens greb, ved at lave et pose  med hånden, hvorpå Myo båndet er. \\
			initiation:	& Bruger\\
			Aktør: & Bruger \& CrustCrawler\\
			Prækondition: & Bruger har Myo bånd på, som er forbundet til pc med genkendelsessoftware kørende \\
			Postkondition: & CrustCrawleren greb er lukket \\
			Hovedscenarie: & 1: Bruger laver pose for at lukke CrustCrawlers greb. \\
			& 2: Systemet genkender posen tilknyttet CrustCrawlerens funktion til at lukke grebet.\\
			& 3: CrustCrawleren lukker grebet.\\

		\end{tabular}
		\caption{Use Case 5}
	\end{table}
\end{center}

\subsection{Åbne CrustCrawler greb}
\begin{center}
	\begin{table}[htbp]
		\rowcolors{2}{white}{lightgrey}
		\begin{tabular}{lp{10cm}}
			\rowcolor{grey} Use case 7:		& Åbne CrustCrawler greb\\
			Mål: & Som Bruger vil man åbne CrustCrawlerens greb, ved at lave et pose  med hånden, hvorpå Myo båndet er.\\
			initiation:	& Bruger \\
			Aktør: & Bruger \& CrustCrawler\\
			Prækondition: & Bruger har Myo bånd på, som er forbundet til pc med genkendelsessoftware kørende. \\
			Postkondition: & CrustCrawleren greb er åbnet. \\
			Hovedscenarie: & 1: Bruger laver pose til at åbne CrustCrawlerens greb\\
			& 2: Systemet genkender posen tilknyttet CrustCrawlerens funktion til at åbne grebet.\\
			& 3: CrustCrawleren åbner grebet.\\
		\end{tabular}
		\caption{Use Case 7}
	\end{table}
\end{center}

\egroup