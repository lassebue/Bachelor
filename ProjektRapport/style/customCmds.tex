%%%%%%%%%%%%%%%%%%%%%%%%%%%%%%%%%%%%%%%%%%%%%%%%%%%%%%%%%%%%%%%%%%%%%%%%%%%%%%%%
%                           Custom Commands                                    %
%                                                                              %
%%%%%%%%%%%%%%%%%%%%%%%%%%%%%%%%%%%%%%%%%%%%%%%%%%%%%%%%%%%%%%%%%%%%%%%%%%%%%%%%

%%%%% FIGURES %%%%%
% The following commands are defined in: style/figure

% Insert a new figure:
% \myFigure{filename}{caption}{label}{width}
% width is ratio of textwidth, so it should be between 0.1 and 1

% Insert a figure wraped in text (left/right)
% \myWrapFigure{filename}{caption}{label}{width}{l/r}
% width is ratio of textwidth, so it should be between 0.1 and 1
% l/r: l = left, r = right

% Insert two figures side by side, with a caption covering both and a subcaption for each figure. OBS: scaled for 50 % text width both
% Command \mySubFigure{filename1}{filename2}{caption}
% {subcaption1}{subcaption2}{label}{sublabel1}{sublabel2}


%%%%% BibTeX %%%%%
% To cite some article use:
% \citep{fooArticle} or
% \citep[see this baws article]{fooArticle}

% OBS: remember to add article to "bibliografi.bib" using syntex of the examples in the file.


%%%%% Nomenclature %%%%%
% When introducing a new symbol add it to the nomenclature by:
% \nomenclature{symbol}{description}
% remember to put symbol in mathmode if necessary
% OBS: if you need to add more the one symbol you have to add a "%" at the end of every \nomenclature use except for the last one. See example below:
%
%	\begin{equation}
%		e=m \cdot c^2
%	\end{equation}%
%	\nomenclature{$e$}{Energy}%
%	\nomenclature{$m$}{The mass}%
%	\nomenclature{$c$}{The speed of light. 299 792 458 $\frac{m}{s}$}


%%%%% TEXT COMMANDS %%%%%
\newcommand{\AD}{Analog Devices }
\newcommand{\sharc}{SHARC }
\newcommand{\board}{EVAL-21469-EZLITE }
\newcommand{\kit}{ADSP-21469 EZ-Kit Lite }


%%%%% LIST COMMANDS &&&&&
\newenvironment{myItemize}
{ \begin{itemize}
	\setlength{\itemsep}{0pt}
	\setlength{\parskip}{0pt}
	\setlength{\parsep}{0pt}     }
{ \end{itemize}                  } 

\newenvironment{myEnumerate}
{ \begin{enumerate}
	\setlength{\itemsep}{0pt}
	\setlength{\parskip}{0pt}
	\setlength{\parsep}{0pt}     }
{ \end{enumerate}                }