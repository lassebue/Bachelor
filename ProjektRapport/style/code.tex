% EXAMPLE
%\begin{lstlisting}[language=Matlab, caption={Dette er et eksempel på hvordan du skal bruge listing}, label=list_me_like_a_baws]
%
%	INSERT CODE HERE
%
%\end{lstlisting}




% Package for code syntax
\usepackage{listings}

\renewcommand{\lstlistingname}{Code listing}

% Package for editing captions
\captionsetup[lstlisting]{
    labelfont=bf,
    justification=raggedright,
    singlelinecheck=false
}



% General settings for code listing
\lstset{
    basicstyle=\footnotesize,       % code font size
    numbers=left,                   % where to put the line-numbers
    numberstyle=\tiny\color{gray},  % line-number style and color
    stepnumber=1,                   % line-numbering steps. 1 = linenumber on every line
    numbersep=5pt,                  % how far the line-numbers are from the code
    backgroundcolor=\color{white},  % the background color.
    showspaces=false,               % show spaces adding particular underscores
    showstringspaces=false,         % underline spaces within strings
    showtabs=false,                 % show tabs within strings adding particular underscores
    frame=single,                   % adds a frame around the code
    rulecolor=\color{black},        % if not set, the frame-color may be changed on line-breaks within not-black text (e.g. commens (green here))
    tabsize=4,                      % sets default tabsize to spaces
    captionpos=top,                 % sets the caption-position to bottom
    breaklines=true,                % sets automatic line breaking
    breakatwhitespace=false,        % sets if automatic breaks should only happen at whitespace
    title=\lstname                  % show the filename of files included with \lstinputlisting
}