\thispagestyle{fancy}
\chapter{Problemformulering}
\label{chp:problemformulering}

Det antages i projektet, at der findes mange anvendelsesmuligheder for genkendelse af håndbevægelser. Blandt andet inden for de tidligere nævnte områder, nemlig VR, myoelektriske proteser, exoskeletter og robotstyring. Her findes der på nuværende tidspunkt en række problemstillinger, men findes der allerede eksisterende løsninger til disse?\\

Ser man på proteser, findes der allerede løsninger, hvor der anvendes EMG til styring af de kunstige lemmer, men her genkendes ikke, hvilke bevægelser brugeren laver. I stedet "bevæger"' brugeren særlige muskler, hvorpå musklens aktivitet registeres. Gentagne "bevægelser"' af en bestemt muskel vil så være en bestemt kommando, der får protesen til at lave en bevægelse. Her vil det dog kun være muligt at aktivere få af protesens bevægelser ad gangen vha. EMG, da der kun kan laves få kommandoer med musklerne. \\

Dette problem har man løst ved at give mulighed for udskifte protesebevægelserne til de enkelte kommandoer. Man kan således vælge, hvilke bevægelser man har brug for på bestemte tidspunkter. I dag kan man vælge protesebevægelserne til hver muskel-kommando på en smartphone app. Ligeledes kan selve protesen styres direkte fra smartphone appen \citep{RefWorks:12}.\\
Dog findes et par problemer med denne løsning, først og fremmest problemet med de få (4)\todoall{ Det er vist 4, men tjek det!!!} kommandoer, hvor langt flere kommandoer vil være nyttige for brugerne. Dog vil ekstra kommandoer nedsætte brugervenligheden, da brugerne vil skulle huske hver kommando og de forskellige kombinationer af muskelspændinger for at lave dem.\\

Videre ses det som et problem, at kommandoerne er nødvendige. Det endelige mål for en armprotese må være, at brugerne i så vid udstrækning som muligt kan bruge proteserne, som om de har rigtige hænder. Her anses kommandoerne for at være en hindring for den intuitive anvendelse af proteserne.\\

Det er her mønstergenkendelse vil kunne anvendes til at genkende de bevægelser, f.eks. at lukke eller åbne hånden, brugerne ville lave med en rigtig hånd. Når bevægelsen er genkendt, vil protesen efterfølgende kunne lave samme bevægelse. Dette vil kun være muligt, hvis man har øvre del af underarmen. \\

Til netop dette formål findes et relativt billigt wearable device kaldet en Myo, der er i stand til at genkende håndbevægelser. Myo er et armbånd med 8 EMG sensorer og 9 aksial IMU med accelerometer, gyroscope og magnetometer, der kan mål EMG-signal og omsætte dem til poses, som bæreren af armbåndet laver. \\
Denne funktionalitet er et stort skridt i den rigtige retning. Brugerne af en armprotese med disse egenskaber vil kunne bruge armprotesen langt mere intuitivt og uden at skulle tænke over, hvilke muskler, der skal spændes for at lave en bestemt kommando.\\
Et problem med Myo'en er dog, at den kun er i stand til at genkende 5 poses. Skulle Myo'en integreres med en armprotese, ville brugerne, således stå i næsten samme situation som med de tidligere nævnte myoelektriske proteser med 4 mulige muskelkommandoer.\\

I forbindelse med exoskeletter forskes, eksperimenteres og udvikles der på Hammel Neurocenter mhp. at finde nye løsninger til genoptræning og rehabilitering af handicappede patienter. Ét af de projekter, der har været igang, er udviklingen af en exoskelet-arm\citep{RefWorks:7}. Denne exo-arm skal hjælpe patienten med at bøje i albueleddet, når det registreres, at biceps-musklen aktiveres. Modsat når triceps-musklen aktiveres, så hjælper exo-armen med at strække patientens arm ud igen. Hensigten med exo-armen er ikke at gøre patienten stærkere, men i stedet at rette fokus på genoptræningen ved at give patienterne evnen til at løfte armen tilbage.\\

På samme måde som exo-armen hjælper med genoptræning i albueleddet, vil anvendelsen af EMG-genkendelsessystemet muligvis kunne hjælpe med genoptræning af hånd og fingre i forbindelse med en exoskelet-hånd. Ved at registrere aktivitet i musklerne i underarmen, vil en exo-hånd muligvis kunne bruges til at hjælpe en patient med at åbne og lukke hånden og derved yde hjælp til den meget vigtige funktion at gribe om og holde fat i genstande og derved forhåbentlig være til hjælp ved genoptræningen og i forsøget på at give patienten førligheden tilbage.

Genkendelse af håndbevægelser vil også kunne anvendes i et virtuelt miljø, hvor man vha.  genkendelsesmodeller vil kunne være i stand til at manipulere med elementer i det virtuelle miljø mere intuitivt vha. hænderne. Et eksempel på dette kunne være et spil, hvor man kan gribe, løfte, flytte rundt på, og give slip på nogle klodser.
