\thispagestyle{fancy}
\chapter{Systembeskrivelse}
\label{chp:systembeskrivelse}

Systemet i dette projekt består af flere dele, nemlig et dataopsamlingsprogram, modeltræning og en testapplikation.
\subsubsection{Dataopsamlingsprogram}
Til denne del af projektet anvendes en Myo.
Da der er behov for data specifikt til machine learning, vil der i projektet blive designet et software program med henblik på at opsamle data sendt fra Myo'en, til senere træning af machine learning modeller.\\
Data collection er en database, som her er hos Parse.com\citep{RefWorks:11}, hvor alle de opsamlede data ligger. Disse kan anvendes til forskellige formål.

\subsubsection{Modeltræning}
Til træning af machine learnings modellerne, er det intentionen at anvende MATLAB indbyggede toolboxes, der er særligt egnet til at udvikle og træne genkendelses-modeller, samt til at teste disse. 
Her er det ligeledes muligt, at undersøge, hvilke typer af machine learning modeller, der er de mest effektive til at genkende poses, og hvilke poses, de forskellige modeller har sværest ved er se forskel på.

\subsubsection{Testapplikation}
I denne del af systemet anvendes de trænede data. En bruger med en Myo kan her lave poses, som sendes til MATLAB-algoritmen, og her genkendes det specifikke pose. Efterfølgende vil anvendelsesprogrammet ved hjælp af den genkendte pose, sende en kommando ud til en en robotarm. Systemet gør brug af en robotarm kaldet CrustCrawler. Denne robot er en arm med 2 bøjelige led, 2 roterende led og en klo, der kan gribe genstande. CrustCrawleren vil modtage en kommando fra testapplikationen, og reagere herpå. Kloen kan lukke eller åbne alt efter om brugeren knytter sin hånd eller åbner den. De poses, der vil blive arbejdet med og indsamlet data på, er dermed:
\begin{myItemize}
	\item Åben hånd - hvor fingrene er strakte
	\item Lukket hånd - hvor hånden er knyttet
	\item Afslappet hånd - hvor hånden er i hvile
\end{myItemize}


På figur \ref{fig:koncept} ses en skitse over systemet. Delen til venstre viser indsamling af data fra Myo'en til en computer, hvorfra der kan uploades til en data collection, som er en database.  Dataene i den online data collection bliver trænet i MATLAB og er derfra klar til anvendelse. På højre side i figuren ses det, at Myo'en kan bæres og bruges til at styre CrustCrawleren vha. MATLAB-algoritmen.

\myFigure{konceptbillede}{Systemskitse, hvor opdeling i de 3 dele ses. Dataopsamling i venstre side fra Myo'en til Data Collection. I midten ses modeltræning, hvor de opsamlede data bruges. til højre ses anvendelsen af modellerne}{fig:koncept}{0.75}

\thispagestyle{fancy}
\section{Funktionelle krav}
\label{sec:funktionellekrav}
I det følgende afsnit vil de funktionelle krav for dataopsamlingsprogrammet og testapplikationen blive fremsat ud fra systembeskrivelsen og de forskellige aktører, som er beskrevet i sektion \ref{sec:aktorbeskrivelser}. Disse krav er opstillet for at målrette det fremadrettede forløb.
Systemets krav bliver opsat i Use Cases, som medvirker til at danne overblik over, hvordan systemets krav bliver opfyldt, og hvordan aktører relaterer til systemet. Alle Use Cases kan ses i kravspecifikationen \citep{RefWorks:8}.

\subsection{Aktørbeskrivelser}
\label{sec:aktorbeskrivelser}
De tre aktører, Testperson, Bruger og CrustCrawler, er beskrevet herunder.

\subsubsection{Testperson}
I systemet bruges en testperson, som har en Myo på. Han anvender dataopsamlingsprogrammet til at opsamle, katagorisere og gemme ny data.

\subsubsection{Bruger}
Brugeren har en Myo på og bruger genkendelssoftwaren til at styre en CrustCrawler-robot med.

\subsubsection{CrustCrawler}
CrustCrawleren er en robotarm med en gribeklo. Denne bruges som en anvendelse til testapplikationen, hvor den reagere på forskellige poses, der kan laves af brugeren.

\subsection*{Tabel over funktionelle krav}
I tabel \ref{tab:funktionellekrav} ses alle de funktionelle krav, der stilles til dataopsamlingsprogrammet og testapplikationen.
\bgroup
\def\arraystretch{1.8}
\begin{center}
	\rowcolors{2}{white}{lightgrey}
	\begin{table}
		\begin{tabular}{lp{225pt}}
			\rowcolor{grey} En testperson skal kunne: &\\
			Krav 1:& Se vejledning til, hvordan program bruges\\
			Krav 2:& Optage datasæt fra Myo\\
			Krav 3:& Tilføje ny pose\\
			Krav 4:& Gemme datasæt\\
			Krav 5:& Uploade datasæt til en data collection\\
			\rowcolor{grey}En Bruger skal kunne: &\\
			Krav 6:& Genkende en pose vha. genkendelsessoftwaren\\
			Krav 7:& Lave lukket hånd, så CrustCrawler-kloen lukker sammen\\
			Krav 8:& Lave åben hånd, så CrustCrawler-kloen åbnes
		\end{tabular}
		\caption{Her ses systemets funktionelle krav. Kravene som en testperson skal kunne, implementeres i dataopsamlingsprogrammet. Kravene som en bruger skal kunne, implementeres i testapplikationen.}
		\label{tab:funktionellekrav}
	\end{table}	
\end{center}
\egroup

\subsection*{Use Case beskrivelser}
For at kunne dække de forskellige funktionelle krav opstilles der en række Use Cases. Her er de forskellige Use Cases gennemgået og beskrevet.

\subsubsection{Se vejledning}
Her skal testpersonen have mulighed for at få en hurtig vejledning i, hvordan dataopsamlingsprogrammet virker.

\subsubsection{Indsaml datasæt og tilføj ny pose}
Testpersonen vælger en pose og indsamler et nyt datasæt, såsom lukket hånd. Herefter kan det uploades til den fælles data collection.
\subsubsection{Genkendelse af pose}
Brugeren skal med testapplikationen genkende en bestemt pose, som Bruger laver med Myo'en på.
\subsubsection{Åbn og luk CrustCrawler greb}
Brugeren skal kunne lade CrustCrawlerens greb åbne og lukke på kommando efter, at man laver knyttet hånd eller åben hånd. Det er her, at anvendelsen er i brug.

\section{Ikke funktionelle krav}
\label{sec:ikkefunktionellekrav}
Systemets ikke funktionelle krav kan findes i \textit{Electromyographic Recognition Using Machine Learning Kravspecifikation}\citep{RefWorks:8}