\thispagestyle{fancy}
\chapter{Systembeskrivelse}
\label{chp:systembeskrivelse}

Systemet i dette projekt består af flere dele, nemlig Dataopsamlingsprogram, Data collection og en Anvendelsesdel.
\subsubsection{Dataopsamlingsprogram}
Til denne del af projektet anvendes et Myo armbånd. Myo Armband er et relativt billigt wearable device som har en række funktionaliteter, hvor den ene er at den kan opfange EMG signaler i underarmen, og sende den opsamlede data til et program.\\
Da der er behov for data specifikt til machine learning, vil der i projektet blive designet et software program, med henblik på at opsamle og ordne data sendt fra Myo båndet, til senere træning af machine learning modeller.


\subsubsection{Data collection}
Data collection, som er anden del af systemet, hvilket her er hostet hos parse.com, er en database hvor alle de opsamlede data ligger. Disse kan anvendes til forskellige formål. Dataene bliver processeret i matlab algoritmer, hvor de bliver trænet til at genkende poses.

\subsubsection{Anvendelse}
I denne del del af systemet anvendes den trænede data. En bruger med en Myo kan her lave poses som sendes til matlab algoritmen og her genkendes det specifikke pose. Efterfølgende vil anvendelsesprogrammet ved hjælp af den genkendte pose, sende en kommando ud til en en robotarm. Systemet gør brug af en robotarm kaldet CrustCrawler. Denne robot er en arm med 2 bøjelige led, 2 roterende led og en klo der kan gribe ud efter genstande. CrustCrawleren vil modtage en kommando fra anvendelsesprogrammet, og reagere herpå. Kloen kan lukke i eller åbne alt efter om brugeren knytter sin hånd eller åbner den.\\


På figur \ref{fig:koncept} ses en skitse over systemet. Delen til venstre er indsamling af data fra Myo armbånd til en computer hvorfra det kan uploades til data collection. Fra data collection ses det at dette er en database. Dataen i data collectionen bliver trænet i matlab og er derfra klar til anvendelse. På højre side ses det at Myoen kan bæres og bruges til at styre crustcrawleren vha. matlab algoritmen.

\myFigure{konceptbillede}{Systemskitse}{fig:koncept}{0.75}