\thispagestyle{fancy}
\chapter{Systembeskrivelse}
\label{chp:systembeskrivelse}

Systemet består af flere dele, hvoraf den ene del er et program der skal opsamle data fra en testperson. Dette gøres med et device kaldet et Myo Armband. Dette er et wearable device, der kan opfange EMG signaler i din underarm, når der laves en pose. Testpersonen kan så registrere en pose, og vælge om hans data skal gemmes og om den skal op i data collection i skyen, hvor andre også ville kunne anvende den.

Data collection, som er anden del af systemet, hvilket her er hostet hos parse.com, er en database hvor alle de opsamlede data ligger. Disse kan anvendes til forskellige anvendelsesformål. På disse data er der arbejdet med machine learning sådan at forskellige poses kan registres og genkendes  når en Mya armband bæres.

Den sidste del af systemet er anvendelsesdelen. Her kan en bruger med et myo armband og en CrustCrawler, få kloen til at gribe ud, ved at lave en knyttet næve. Dette viser hvordan vi ved hjælp at EMG signaler og machine learning er i stand til at styre en robot arm.\todoall{Matlab?}\\
På figur \ref{fig:koncept} ses en skitse over systemet. Delen til venstre er indsamling af data fra Myo armband til en computer hvorfra det kan uploades til data collection. Fra data collection ses det at dette er en database og videre er det i anvendelsen i model træning at dataen fra databasen bliver anvendt. her bliver det processeret i matlab, der beregner hvilken pose der bliver sendt fra Myo'en, ved at sammenligne data med måling. Herefter gives en kommando ud til CrustCrawleren, der reagerer.

\myFigure{konceptbillede}{Systemskitse}{fig:koncept}{0.75}