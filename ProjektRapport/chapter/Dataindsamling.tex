\thispagestyle{fancy}
\chapter{Dataindsamling}
\label{chp:dataingsamlingChapter}
I sektion \ref{sec:fysiologi} er det beskrevet, hvordan der kan opsamles data fra aktivitet i musklerne ved hjælp af elektromyografi. For at disse data kan opsamles, er der blevet udviklet et værktøj, så det er muligt for alle at opsamle data og uploade det til data collection, hvormed alle kan få gavn af de opsamlede data. Data Collection Application var hovedfokus i de første uger af projektet, da det skulle bruges i projektet, for at opsamle data, der senere skulle bearbejdes og trænes med machine learning. Applikationen er udviklet ud fra de krav, der er opsat i kravspecifikationen\citep{RefWorks:8}. I applikationen skal det altså være muligt at:
\begin{myItemize}
	\item Se instruktion for brug af applikation
	\item Check orientering
	\item Angiv hånd
	\item Angiv Testperson
	\item Indsamle datasæt
	\item Tilføje ny pose
\end{myItemize}
Instruktionen er en forklaring på de forskellige knapper på GUI'en. Den giver et overblik og en forståelse for, hvordan applikationen betjenes. Denne kan åbnes fra "File"' oppe i venstre hjørne af GUI'en. "File"' ses på figur \ref{fig:dca}.

\section{Optagelse af data}
\label{sec:optagelsedata}
For hvert enkelt sample, der optages, samles 9 informationer. Den første information er tiden, der er gået siden start af optagelsen. De næste 8 informationer er én integer for hver af de 8 EMG-sensorer, der sidder på Myo'en. 3 EMG-sensorer kan ses tilbage på figur \ref{fig:myo}.
På figur \ref{fig:sample} ses et eksempel på et udsnit fra en sekvens af samples.

\myFigure{sample2}{Eksempel fra en sekvens af samples, hvor informationerne, der optages, kan ses. time er tiden, der er gået siden starten af optagelsen. emg1-8 er EMG-signalerne fra de 8 sensorer som integers. hand er den hånd, der er indsamlet data fra. pose er posen, der opsamles data for. Dette er et PoseID, der kan variere fra 0 til 2. orientation er Myo'ens placering. Denne går fra 0 til 49 og fortæller, hvordan Myo'en sidder på armen. testPerson, er den person der har opsamlet dataene.}{fig:sample}{1}

Som det ses, er der på første sample flere informationer. Grunden til dette er, at de ikke ændres i løbet af én optagelse. Posen der optages, gemmes da dataopsamlingen er input og pose er output. Orientering af Myo'en gemmes, da den kan variere, idet den ikke nødvendigvis tages på armen på samme måde hver gang. hand bliver gemt, men da det er besluttet kun at bruge højre hånd i projektet, bliver denne ikke videre behandlet.\\
Data bliver optaget ved, at Myo'en påføres underarmen, hvor den er tykkest, som det ses på figur \ref{fig:myopic}. Herefter checkes orienteringen ved at holde armen strakt ud med tommefingeren op. Den ønskede pose, testpersonen vil optage, vælges i fold-ud menuen, og optagelsen startes. På figur \ref{fig:dca} ses brugergrænsefladen til applikationen.

\myFigure{myopic}{Myo'en, placeret hvor underarmen er tykkest, klar til at opsamle data.}{fig:myopic}{0.4}

\myFigure{dca}{Data Collection Applikation brugergrænseflade. Her ses de omtalte knapper, check orientation, hvor orienteringen af Myo'en kan findes. Der kan også tilføjes en ny pose til databasen ved klik på "Add New Pose"'. Brugernavn skal indtastes for at kunne optage, så det er muligt at se på databasen, hvem der har uploaded hvad.}{fig:dca}{0.9}

Når en optagelse startes, laves den ønskede pose, og der registreres EMG-signaler fra musklerne med Myo'en. Signalerne sendes som integers til applikationen, der skriver dem ud i en .csv-fil. Når en optagelse stoppes, bliver .csv-filen gemt, og der er mulighed for at få gemt data online. Gemmes data online bliver det sendt til en database på Parse.com \citep{RefWorks:11}. Her bliver alle data gemt fra testpersoner, der anvender applikationen.

Til opbevaring af pose-data er der valgt .csv-fil. Valget stod mellem en database eller .csv-filen. .csv-filen kan læses i en bred udstrækning af miljø og applikationer, og er nem at arbejde med. Det er derfor en smart løsning, at hver optagelse ligger i en fil for sig. Derfor blev .csv valgt. Filerne er samlet på Parse.com i en database. Hos Parse.com er der så mulighed for at hente .csv-filerne ud og arbejde med dem lokalt. Dette er f.eks. til træning af machine learning modeller.

\section{Delkonklusion}
Med dette værktøj, er det muligt at opsamle data og gemme det lokalt og på Parse.com. Dataene kan importeres i MATLAB, hvor der kan laves feature extraction og modeltræning af machine learning modeller. Alle kan, hvis de er online, opsamle data og lægge det op på Parse.com. En samlet database med alle data kan findes, downloades og anvendes til træning af machine learning modeller.