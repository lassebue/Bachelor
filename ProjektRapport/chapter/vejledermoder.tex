\chapter{Vejledermøder}
\label{bilag:vejl}
\vspace{-1cm}
\begin{center}
	\begin{tabular}{| l | p{10cm} |}
		\hline
		Dato		& 9. juli 2015\\ \hline
		tilstede 	& Lasse bue Svendsen, Kristoffer Sloth Gade og Peter Ahrendt\\ \hline
		Referat		& \vspace{-5mm}\begin{myEnumerate}
			\item Fortæl om vores idé
			\begin{myItemize}				
				\item Dataopsamlingsprogram
				\item ML algoritmer og undersøgelse af dem , forskellige ML muligheder
				\item Generel undersøgelse af genkendelsesmuligheder
				\item Anvendelse
				\begin{myItemize}
					\item Unity
					\item Robotarm styring
				\end{myItemize}
			\end{myItemize}
			\item Udvikling
			\begin{myItemize}
				\item C\#
				\item Matlab
			\end{myItemize}
			\item Struktur i projekt
			\begin{myItemize}
				\item Ugentlig vejledermøde
				\item Vi indkalder med dagsorden
				\item T-Model
				\begin{myItemize}
					\item Skal lave et helt system
					\item gå i dybden med én ting
				\end{myItemize}
				\item Samlet præsentation
				\begin{myItemize}
					\item Derefter ind hver især
				\end{myItemize}
				\item God ide med midtvejs rapport
				\begin{myItemize}
					\item Første version
					\item aflever rapport på det
				\end{myItemize}
				\item typisk skrive rapport de sidste to uger
			\end{myItemize}
			\item Myo armbånd og opsamling 
			\begin{myItemize}
				\item Peter forsøger at argumentere for indkøb af Myo
				\item Forstå hvordan data bliver opsamlet
				\item Hvad er princippet i målingen
			\end{myItemize}
			\item Problemformulering
			\begin{myItemize}
				\item Lav en klar problemformulering
				\begin{myItemize}
					\item Lav en klar prblemformulering
					\item hvad vil vi gerne gøre
					\item Vælg også bevægelse
				\end{myItemize}
			\end{myItemize}
			\end{myEnumerate}\\  	
		\hline
	\end{tabular}
\end{center}

\begin{center}
	\begin{tabular}{| l | p{10cm} |}
		\hline
		Dato		& 10. septembe 2015\\ \hline
		tilstede 	& Lasse bue Svendsen, Kristoffer Sloth Gade og Peter Ahrendt\\ \hline
		Referat		& \vspace{-5mm}\begin{myEnumerate}
			\item Forberedelse
			\begin{myItemize}				
				\item Myo på vej, kommer den 17. sep.
				\item Fokus på dataopsamlingsprogram
				\begin{myItemize}
					\item Gennemgår software framework
					\item Kigget på signalbehandling af data
					\item Visning af direkte visualisering, samt matlab data
				\end{myItemize}
				\begin{myItemize}
					\item Kompatibel med alt
				\end{myItemize}
			\end{myItemize}
			\item Klar problemformulering
			\begin{myItemize}
				\item Opsamlingsprogram
				\item Machine Learning program
				\item Realtidsanvendelse af algoritme
				\begin{myItemize}
					\item Robotarm
					\item Computer med visualisering af hånd  e.l.
				\end{myItemize}
			\end{myItemize}
			\item Kontinuert eller afskåret pose data?
			\begin{myItemize}
				\item Vi antager det er bedst med et dataset pr. pose
				\item dataset pr. pose er en instans i træningssættet
				\item Statistik - for inaktivt signal vs. aktivt signal
			\end{myItemize}
			\item Det er evt. muligt at samarbejde med Hammel Neurocenter
		\end{myEnumerate}\\ 	
		\hline
	\end{tabular}
\end{center}

\begin{center}
	\begin{tabular}{| l | p{10cm} |}
		\hline
		Dato		& 17. September 2015\\ \hline
		tilstede 	& Lasse bue Svendsen, Kristoffer Sloth Gade og Peter Ahrendt\\ \hline
		Referat		& \vspace{-5mm}\begin{myEnumerate}
			\item Matlab Machine Learning
			\begin{myItemize}				
				\item Et godt værktøj
			\end{myItemize}
			\item Anvendelse
			\begin{myItemize}
				\item CrustCrawler er en god anvendelse
				\item evt. Use Cases hertil
			\end{myItemize}
			\item Process
			\begin{myItemize}
				\item Kan nemt dele op i iterationer
			\end{myItemize}
			\item Kravspecifikation
			\begin{myItemize}
				\item Nemmere at finde ud af hvad vi vil
			\end{myItemize}
			\item Midtvejsaflevering
			\begin{myItemize}
				\item 23. oktober
			\end{myItemize}
		\end{myEnumerate}\\	
		\hline
	\end{tabular}
\end{center}

\begin{center}
	\begin{tabular}{| l | p{10cm} |}
		\hline
		Dato		& 15. Oktober 2015\\ \hline
		tilstede 	& Lasse bue Svendsen, Kristoffer Sloth Gade og Peter Ahrendt\\ \hline
		Referat		& \vspace{-5mm}\begin{myEnumerate}
			\item Kravspecifikation
			\begin{myItemize}				
				\item Er blevet sendt til vejleder
			\end{myItemize}
			\item CrustCrawler
			\begin{myItemize}
				\item Mål
				\begin{myItemize}
					\item Åbne klo
					\item Lukke klo
				\end{myItemize}
				\item Kan CrustCrawler styres fra Windows
				\begin{myItemize}
					\item Driver kan findes på Dynamixel
					\begin{myItemize}
						\item Software
						\item Matlab
					\end{myItemize}
				\end{myItemize}
			\end{myItemize}
			\item Midtvejsaflevering
			\begin{myItemize}
				\item Udskydes til fredag den 30. okt. pga. eksaminer
			\end{myItemize}
			\item Dataopsamling
			\begin{myItemize}
				\item Fortælle status herpå
			\end{myItemize}
		\end{myEnumerate}\\ 	
		\hline
	\end{tabular}
\end{center}

\begin{center}
	\begin{tabular}{| l | p{10cm} |}
		\hline
		Dato		& 23. Oktober 2015\\ \hline
		tilstede 	& Lasse bue Svendsen, Kristoffer Sloth Gade og Peter Ahrendt\\ \hline
		Referat		& \vspace{-5mm}\begin{myEnumerate}
			\item Rapport
			\begin{myItemize}				
				\item Vi føder der er okay styr på rapporten
				\item Sørg for den røde tråd
				\item Vise at vi har forstået, ikke  kun det vi selv laver
				\item Hvilke udfordringer er der hen til det mål vi vil, også for det andre har lavet
				\item Problemet? Hvilke andre muligheder kunne man have valgt. hvad er realistiske krav?
				\item Relation til andres idéer
				\item Se på hvad andre har lavet
				\begin{myItemize}
					\item Afgrens og lav krav
				\end{myItemize}
				\item Behøver ikke være så “Krav firkantet” Der er ikke en reel kunde
				\item Dokumentation skal ikke nødvendigvis være der eller være kæmpe
				\item Dokumentation kan også lægges i bilag i rapporten
				\item Hvad har man i rapporten selv lyst til at læse
				\item Test kan bare være grøn lampe blinker ex. fra peter		
			\end{myItemize}
			\item Kravspecifikation
			\begin{myItemize}
				\item Det tilfredsstillende krav? Nok materiale osv.?
				\begin{myItemize}
					\item Ja det er nok, bare husk kvalitet
				\end{myItemize}
			\end{myItemize}
			\item Accepttest
			\begin{myItemize}
				\item Kun hvis det er nødvendigt
				\item undersøgelses projekt er anderledes
			\end{myItemize}
			\item Systemarkitektur
			\begin{myItemize}
				\item Hvordan skal det opdelingen være mht. dataopsamling, modeltræning, genkændelse og CrustCrawler styring?
				\item det vigtige er den røde tråd
			\end{myItemize}
			\item Øvrig dokumentation
			\begin{myItemize}
				\item Kommer ikke med
			\end{myItemize}
			\item Eksamen
			\begin{myItemize}
				\item Hvorfor gjorde i som i gjorde
				\begin{myItemize}
					\item andre prøvede og det virkede/virkede ikke
				\end{myItemize}
			\end{myItemize}
		\end{myEnumerate}\\ 	
		\hline
	\end{tabular}
\end{center}

\begin{center}
	\begin{tabular}{| l | p{10cm} |}
		\hline
		Dato		& 5. November 2015\\ \hline
		tilstede 	& Lasse bue Svendsen, Kristoffer Sloth Gade og Peter Ahrendt\\ \hline
		Referat		& \vspace{-5mm}\begin{myEnumerate}
			\item Midtvejrapport feedback
			\begin{myItemize}				
				\item Krav ikke kapiel for sig. måske under analyse
				\item Læringsmål behøver ikke være med
				\begin{myItemize}
					\item Evt. underafsnit i problemformulering
				\end{myItemize}
				\item Skal noget laves om
				\begin{myItemize}
					\item Afsnit med problemformulering
					\begin{myItemize}
						\item Skal det evt. være et andet sted
					\end{myItemize}
					\item Der kommer mere ift. Machine Learning
				\end{myItemize}
				\item Hvor lang skal rapporten være
				\begin{myItemize}
					\item Over 50 sider
					\item aflevér 3 eks. + 3 cd'er
				\end{myItemize}
			\end{myItemize}
			\item Status
			\begin{myItemize}
				\item CrustCrawler
				\begin{myItemize}
					\item Styring gennem matlab er under udvikling
					\item Har haft problemer men er på sporet igen
				\end{myItemize}
				\item DCA
				\begin{myItemize}
					\item Orientering gemmes
					\item Bedre model hvis der trænes med den
					\item Undersøgelse om armbåndets placering på armen
					\item Gemmer data online
				\end{myItemize}
				\item Genkendelse
				\begin{myItemize}
					\item Vi kan genkende realtid fra Myo'en
				\end{myItemize}
			\end{myItemize}
			\item Hvad kan vi undersøge/eksperimentere med?
			\begin{myItemize}
				\item Kende forskel mellem personer
				\item Armbåndsorienteringens påvirkning af genkendelse
				\item Det er bedst at gå i dybden med få undersøgelser i stedet for at lave mange forskellige
				\item vigtigt at vise vi har forstået det er er i rapporten
			\end{myItemize}
		\end{myEnumerate}\\ 	
		\hline
	\end{tabular}
\end{center}

\begin{center}
	\begin{tabular}{| l | p{10cm} |}
		\hline
		Dato		& 13. November 2015\\ \hline
		tilstede 	& Lasse bue Svendsen, Kristoffer Sloth Gade og Peter Ahrendt\\ \hline
		Referat		& \vspace{-5mm}\begin{myEnumerate}
			\item Individuel fokus i rapport og tileksamen
			\begin{myItemize}				
				\item Brug ikke for lang tid, censor vil gerne snakke
				\item Det vi præsenterer bliver der oftest stillet spørgsmål
				\item 10-15 min sammen inde
				\item Alene
				\begin{myItemize}
					\item 10 min præsentation
					\item ca. 30 min spørgsmål
				\end{myItemize}
				\item Tænk over hvem der skal lave og gå i dybden med hvad
				\item Fokus kunne evt. Være hvordan kan Matlab integrationen kører så tæt på realtid som muligt
			\end{myItemize}
			\item Fysiologi i rapporten
			\begin{myItemize}
				\item Fint med ikke for meget i dybden
				\item Godt i forhold til Myo og hvorfor det giver mening
				\item Hurtigt skal man i rapporten vide hvad det er vi har lavet af det hele
			\end{myItemize}
			\item Teoriafsnit om Machine Learning
			\begin{myItemize}
				\item Skriv om de ting vi har sat os ind i og forstået
				\begin{myItemize}
					\item Hvad skal man bruge det til
					\item Hvordan bruges det
					\item Hvad bruges det til nu
				\end{myItemize}
			\end{myItemize}
		\end{myEnumerate}\\ 	
		\hline
	\end{tabular}
\end{center}

\begin{center}
	\begin{tabular}{| l | p{10cm} |}
		\hline
		Dato		& 26. November 2015\\ \hline
		tilstede 	& Lasse bue Svendsen, Kristoffer Sloth Gade og Peter Ahrendt\\ \hline
		Referat		& \vspace{-5mm}\begin{myEnumerate}
			\item Machine Learning afsnit
			\begin{myItemize}				
				\item er det fint at gå ind og redegøre for de forskellige modeller
				\begin{myItemize}
					\item God ide med redegørelse af modeller
					\item Er det svært at forstå og vi ikke helt forstår det, så lad være med at ta det med
					\item Sammenligning af forskellige metoder
				\end{myItemize}
				\item Evt. sammenligne gode og dårlige
			\end{myItemize}
			\item er det god ide med manual til DCA og dokumentation til CCA + evt. manual
			\begin{myItemize}
				\item God ting at ha manual, men ikke lægge kræfter i det, ikke mere end en halv dag
			\end{myItemize}
			\item Hvordan kan Matlab indgå i rapporten
			\begin{myItemize}
				\item Gør meget som et hvert andet prog. Sprog
				\item Man kan godt tage 3 linjer ud og forklare at sådan gør vi dette
				\item Databehandling
				\item Modeltræning
				\item Test af model
			\end{myItemize}
			\item Undersøgelse af resultater efter kun optagelse i én position
			\begin{myItemize}
				\item ville det være godt at have med ?
				\item Sammenligne med Myo
			\end{myItemize}
		\end{myEnumerate}\\ 	
		\hline
	\end{tabular}
\end{center}

\begin{center}
	\begin{tabular}{| l | p{10cm} |}
		\hline
		Dato		& 10. December 2015\\ \hline
		tilstede 	& Lasse bue Svendsen, Kristoffer Sloth Gade og Peter Ahrendt\\ \hline
		Referat		& \vspace{-5mm}\begin{myEnumerate}
			\item Feedback på rapport
			\begin{myItemize}
				\item Generelt ser det godt ud
				\item "Matlab"' kan ikke bare være en overskrift
			\end{myItemize}
			
		\end{myEnumerate}\\ 	
		\hline
	\end{tabular}
\end{center}