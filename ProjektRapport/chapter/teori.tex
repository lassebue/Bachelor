\thispagestyle{fancy}
\chapter{Teori}
\label{chp:teori}

\section{Fysiologi}
\label{sec:fysiologi}
I projektet er der brugt den tidligere omtalte Myo, fra Thalmic Labs. Myo'en bliver brugt til at optage data fra testpersoner. Dataene kan bruges til at træne machine learning modeller og anvende disse til udvikling af applikationer, der intuitivt kan reagere på elektromyografiske signaler fra armen.
I dette kapitel vil der blive givet et overblik over, hvordan aktiviteten i musklerne registreres. Der vil specifikt blive kigget på musklerne i underarmen, da det er signalerne herfra, der arbejdes med i projektet.


\myWrapFigure{myo}{Indersiden af Myo'en, hvor sensorerne har grænseflade til huden på underarmen. Myo'en placeres, hvor underarmen er tykkest for at få  det bedst mulige resultat. Der ses 3 pods, som hver er ens. Hver pod består af en EMG-sensor. På den midterste ses to store elektroder til at opfange signaler, mens den smalle i midten er til reference.}{fig:myo}{0.5}{r}

Når mennesket bevæger sig, sendes der signaler fra hjernen til rygmarven og herfra videre til de motoriske nerver (bevægenerverne), som til sidst overfører signalerne til musklerne. Musklerne i kroppen bliver aktiveret og trækker sig sammen, når de få signal til det. Det er en kombination af flere muskelgrupper i underarmen, der bevirker, at hånden og fingrene kan bevæges. Når muskelgrupperne på underarmens bøjeside aktiveres, fremkalder det først og fremmest bøje-bevægelser i håndled og fingre. Modsat når muskelgrupperne på underarmens strækkeside aktiveres, så fremkalder det overvejende udstrækning af håndled og fingre. Musklerne er opbygget af mange små muskelfibre. Når disse modtager elektriske signaler fra nerverne, trækker de sig sammen og dermed også hele musklen. Det er netop de elektriske signaler, der forårsager disse sammentrækninger i musklerne, der i projektet bliver opsamlet data på. De elektriske signaler kan opsamles med en Myo. Myo'ens sensorer består hver af en plus-elektrode og en minus- elektrode og en reference-elektrode. Reference-elektroden adskiller plus- og minus- elektroden. Se figur \ref{fig:myo}

De to elektroder måler de elektriske signaler fra muskelfibrene under muskelaktiviteten\citep{RefWorks:13}. Første og anden elektrode på samme sensor sidder placeret uden på huden og kan derfor både sidde på samme muskel, men også hvor muskler overlapper hinanden. Når der opsamles data med Myo'en, bliver EMG-signalerne bearbejdet af Myo'en, der giver en integer som output pr. sample for hver EMG-sensor. Output har ingen enhed, men er en approksimation af aktivitet, der er over sensoren ved sampling. På figur \ref{fig:anatomy} kan det ses, hvordan musklerne sidder i armen. Det er altså muligt at have en EMG-sensor til at sidde delvist på den muskel, der er markeret med gult og delvist på den muskel, der er markeret med grønt.

\myFigure{anatomy}{Bøjeside af venstre underarm med markering af muskler. Det ses, hvordan forskellige muskler hæfter forskelligt i håndleddet og dermed har hver sin funktion i hånden. Den grønne muskel kan være den, der ved sammentrækning bøjer tommelfingeren, mens den blå ved sammentrækning bøjer håndleddet.}{fig:anatomy}{1}

Når Myo'en tages på, kan den sidde på forskellige måder på armen, dog skal den sidde, hvor armen er tykkest. Dette betyder, at en sensor kan sidde på den med rødt markerede muskel én gang, og næste gang den tages på, kan den måske sidde på den med gult markerede muskel. Da musklerne er forbundet til hver sin funktionalitet i hånden, vil der komme to forskellige udfald i de opsamlede data. Der skal derfor tages højde for Myo'ens placering på armen, når der opsamles data.\\
Dette bliver nærmere beskrevet i sektion \ref{sec:optagelsedata}.

\section{Machine Learning}
\label{sec:machineLearning}
I følgende sektion vil den grundlæggende teori for brugen af machine learning blive belyst. Machine learning er læren om, at få computere til selv at "tænke"' uden rent faktisk at programmere dem.

Formålet med machine learning er, at lade programmer lære ud fra opsamlet data, i stedet for at programmere dem. Der er tre typer af machine learning\citep{PatternBishop}:
\begin{description}
	\item[Supervised learning] er, hvor træningsdata-inputtet sammen med det ønskede output, gives til computeren. Her bliver den lært, hvilket output et givet input passer til.
	\item[Unsupervised learning] er derimod, hvor der gives træningsdata, men ikke et tilhørende output. Her er det altså op til computeren at finde et mønster, og dermed at kunne genkende lignende mønstre ud fra træningsdataene.
	\item[Reinforcement learning] er, hvor computeren befinder sig i et dynamisk miljø, og der skal findes løsninger for at komme nærmest målet. Reinforcement learning er bl.a. brugt i selvkørende biler.
\end{description}

I dette projekt er der arbejdet med supervised learning. Der bliver opsamlet data med Myo'en og angivet, hvad inputtet er og dermed også, hvad den skal give af output. Inputtet er et specifikt datasæt, der bliver opsamlet. Den specifikke data skal bruges til at genkende et mønster på nye data. Dette kaldes \textit{induktion}, det specifikke skal bruges til at genkende det generelle. Omvendt er \textit{deduction} at gå den anden vej, fra det generelle til det specifikke.

\subsection{Fremgangsmåde}
Når et problem løses med machine learning, gøres det gennem en række steps. Der skal først loades data, der er blevet opsamlet. Dernæst bliver den opsamlede data præ-processeret. Under præ-processeringen findes dataenes features. En feature er et kendetegner kan trækkes ud af data. Når dataene er præ-processeret, trænes modeller til at genkende, hvilke inputs der mappes til specifikke ouputs. Dette er supervised learning og giver en model, der kan anvendes til mønstergenkendelse. På figur \ref{fig:mlprocess} ses processen fra loading af data til trænet model \citep{MLMadeEasy}.

\myFigure{mlprocess}{I præ-processeringen af dataene bruges feature extraction for at klargøre dataene til træning. Features er elementer der kendetegner dataene. Mere om dette i sektion \ref{sec:featurex}. Herefter er dataene klar til at blive trænet. Med supervised learning, trænes en model ud fra data-input til at genkende output. Præ-processering og træning af model, kan være en iterativ proces, hvor der prøves forskellige machine learning algoritmer}{fig:mlprocess}{1}

Nu er modellen lavet og klar til at lave mønstergenkendelse. Som det ses på figur \ref{fig:mlprocess2}, er det første der sker, det samme som bliver gjort under træning af en ny model. Derefter køres det igennem modellen, som giver resultater på dataene.

\myFigure{mlprocess2}{Der bliver ny data opsamlet, som bliver præ-processeret. Når data er præ-processeret bliver de kørt igennem modellen, der løber alle dataene igennem og laver genkendelse på dem.}{fig:mlprocess2}{1}

\subsection{Problematik}
Når der trænes machine learning modeller, er det ikke altid, at de laver genkendelse korrekt. To problemer man kan støde på ved en model, er \textit{overfitting} og \textit{underfitting}. Overfitting opstår, når en model bliver for specifik og prøver "for meget"' at ramme hver enkel sample i træningssættet.

\subsubsection{Over- og Underfitting}
Hvis der er lavet en kompleks model, der er meget fleksibel og specifik, kan den være overfittet. På figur \ref{fig:over}\citep{fitting} ses et eksempel på en model med overfitting. Umiddelbart ser den ud til at være god, da den går gennem næsten alle samples. Problemet er, at den afviger meget fra den funktion, der er datasamples fra. Derfor kan den lave outputs, der ligger langt fra målet. På figur \ref{fig:under} ses det modsatte eksempel. På denne model ses underfitting. Modellen er en lige linje langs de samplede data, men den misser en masse samples, da den ligger for langt fra hver datasample. Den passer derfor dårligt på funktionen, den skal fitte på.

\mySubFigure{overfitting}{underfitting}{Overfitting og underfitting. Det ses hvordan begge modeller har deres svagheder. Modellen på \ref{fig:over} har stor risiko for at "genkende"' for mange data som sit mål. Dog finder den alle de rigtige. På modellen på \ref{fig:under} derimod, finder den ikke de samples den bør, og genkender dermed ikke de mål den bør. }{}{}{fig:fitting}{fig:over}{fig:under}

\subsection{Løsninger}
For at undgå overfitting skal man sørge for ikke at have for mange features i træningsdataene. De kan evt. fjernes manuelt, så man udvælger, hvilke der skal bruges og hvilke, der ikke skal.
Når der trænes i MATLAB er der valideringsmetoder, der sørger for at reducere overfitting. En metode er \textit{Cross-Validation}. Her deles datasættet op i mindre stykker, hvor én del bruges til test, og resten til træning, se figur \ref{fig:crossval}. Man vælger selv, hvor mange foldninger, der skal bruges. Denne type validering er bedst til mindre datasæt på under 6000 observationer.

\myFigure{crossval}{I dette tilfælde er der valgt 3 foldninger. Datasættet deles derfor op i 3 lige store dele, hvor én af dem bruges til test, mens de to andre bruge til træning. Dette gøres 3 gange, hver gang hvor et nyt subset bruges til test.}{fig:crossval}{0.9}

En anden måde at mindske overfitting på, er ved at bruge \textit{Holdout-validation}. Ved denne validering vælges en procentdel af datasættet, som bruges til test, den øvrige del anvendes som træningssæt. Modellen bliver valideret, ved at teste den trænede model mod testsættet. Holdout-validation er god til store mængder data på over 6000 observationer.

Med disse valideringsmetoder, bliver det muligt at se på modellen, om den er brugbar.