\thispagestyle{fancy}
\chapter{Teori}
\label{chp:teori}

\section{Fysiologi}
\label{sec:fysiologi}
I projektet er der brugt det tidligere omtalte Myo armbånd der er et wearable device fra Thalmic labs. Dette bliver brugt til at optage data fra testpersoner, som kan bruges til at træne machine learning modeller, og anvende disse til udvikling af applikationer. der intuitivt kan reagere på menneskets signaler der bliver sendt ud i musklerne. Men hvordan hænger det sammen mellem musklerne og Myo'en? Dette kapitel vil give et overblik over hvordan der registreres aktivitet i musklerne, i dette tilfælde vil der specifikt blive kigget på musklerne i underarmen da det er signalet herfra der arbejdes med i projektet.

Når mennesket bevæger sig, er det signaler der sendes fra hjernen og ud igennem nerverne til musklerne. Musklerne i kroppen bliver aktiveret når de får signal til det og trækker sig sammen. En kombination af muskelgrupperne i underarmen der gør dette kan få mennesket til at bevæge hånden og fingrene. Det er altså sammentrækninger af muskelgrupper der resulterer i bevægelser. Det er netop de signaler der forsager disse sammentrækninger der i projektet bliver opsamlet data på. Når en muskel trækker sig sammen, er det fordi hjernen sender et elektrisk signal i nerverne igennem rygmarven og ud til en muskel. I musklen er nerven forbundet til alle de små muskelfibre, som alle vil trække sig sammen, hvilket gør at hele musklen trækker sig sammen. Det elektriske signal der trækker musklen sammen, kan opsamles med en Myo. Myo'en består af 8 EMG sensorer der hver er to elektroder, en plus elektrode og en minus elektrode. Derudover er der også en reference elektrode der adskiller de to plus og minus elektroder. se figur \ref{fig:myo}

\myFigure{myo}{Indersiden af Myo'en hvor sensorerne har grænseflade til huden på underarmen. Myo'en placeres hvor underarmen er tykkest for at få bedst muligt resultat. Der ses 3 pods, som hver er ens. Hver pod består af en EMG sensor. På den midterste ses to store elektroder til at opfange signaler mens den smalle i midten er til reference.}{fig:myo}{0.6}

De to elektroder måler de elektriske signaler som muskelfibrene skaber ved muskelaktivitet.
\citep{RefWorks:13}. Første og anden elektrode på samme sensorsidder placeret uden på huden og kan derfor sidder på samme muskel, men også hvor muskler overlapper hinanden. Når der opsamles data med Myo'en, bliver EMG signalen bearbejdet af Myo'en, så den giver en integer som output på hver EMG snesor for hver sample der tages. Denne integer kan der så arbejdes med og der kan med mange samples laves machine learning modeller.

\section{Machine Learning}
\label{sec:machineLearning}

\subsection{Formål}

test set/ validation set
Træningssæt \todola{Der bruges training set, så alle begreber er på engelsk}

\subsection{Proces}
\todola{Hvordan gøres det, beskriv trinene}
\subsection{Problematik}

Kurve med validationsfejl kontra fejl/ fejl på træningssættet. 

\subsubsection{overfitting}
Varians - sensitivitet over for individuelle data punktet i træningssæt, fejl i træningssættet
\subsubsection{underfitting}
