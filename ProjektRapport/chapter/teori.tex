\thispagestyle{fancy}
\chapter{Teori}
\label{chp:teori}

\section{Fysiologi}
\label{sec:fysiologi}
I projektet er der brugt den tidligere omtalte Myo, der er et wearable device fra Thalmic Labs. Myo'en bliver brugt til at optage data fra testpersoner. Dataene kan bruges til at træne machine learning modeller og anvende disse til udvikling af applikationer, der intuitivt kan reagere på elektromyografiske signaler fra armen.
I dette kapitel vil der blive givet et overblik over, hvordan aktiviteten i musklerne registreres. Der vil specifikt blive kigget på musklerne i underarmen, da det er signalerne herfra, der arbejdes med i projektet.


\myWrapFigure{myo}{Indersiden af Myo'en, hvor sensorerne har grænseflade til huden på underarmen. Myo'en placeres, hvor underarmen er tykkest for at få  det bedst mulige resultat. Der ses 3 pods, som hver er ens. Hver pod består af en EMG-sensor. På den midterste ses to store elektroder til at opfange signaler, mens den smalle i midten er til reference.}{fig:myo}{0.5}{r}

Når mennesket bevæger sig, sendes der signaler fra hjernen til rygmarven og herfra videre til de motoriske nerver (bevægenerverne), som til sidst overfører signalerne til musklerne. Musklerne i kroppen bliver aktiveret og trækker sig sammen, når de få signal til det. Det er en kombination af flere muskelgrupper i underarmen, der bevirker, at hånden og fingrene kan bevæges. Når muskelgrupperne på underarmens bøjeside aktiveres, fremkalder det først og fremmest bøje-bevægelser i håndled og fingre. Modsat når muskelgrupperne på underarmens strækkeside aktiveres, så fremkalder det overvejende udstrækning af håndled og fingre. Musklerne er opbygget af mange små muskelfibre. Når disse modtager elektriske signaler fra nerverne, trækker de sig sammen og dermed også hele musklen. Det er netop de elektriske signaler, der forårsager disse sammentrækninger i musklerne, der i projektet bliver opsamlet data på. De elektriske signaler kan opsamles med en Myo. Myo'ens sensorer består hver af en plus-elektrode og en minus- elektrode og en reference-elektrode. Reference-elektroden adskiller plus- og minus- elektroden. Se figur \ref{fig:myo}

De to elektroder måler de elektriske signaler fra muskelfibrene under muskelaktiviteten\citep{RefWorks:13}. Første og anden elektrode på samme sensor sidder placeret uden på huden og kan derfor både sidde på samme muskel, men også hvor muskler overlapper hinanden. Når der opsamles data med Myo'en, bliver EMG-signalerne bearbejdet af Myo'en, der giver en integer som output pr. sample for hver EMG-sensor. Output har ikke ingen enhed, men er en approksimation af aktivitet, der er over sensoren ved sampling. På figur \ref{fig:anatomy} kan det ses, hvordan musklerne sidder i armen. Det er altså muligt at have en EMG-sensor til at sidde delvist på den muskel, der er markeret med gult og delvist på den muskel, der er markeret med grønt.

\myFigure{anatomy}{Bøjeside af venstre underarm med markering af muskler. Det ses, hvordan forskellige muskler hæfter forskelligt i håndleddet og dermed har hver sin funktion i hånden. Den grønne muskel kan være den, der ved sammentrækning bøjer tommelfingeren, mens den blå ved sammentrækning bøjer håndleddet.}{fig:anatomy}{1}

Når Myo'en tages på, kan den sidde på forskellige måder på armen, dog skal den sidde, hvor armen er tykkest. Dette betyder, at en sensor kan sidde på den med rødt markerede muskel én gang, og næste gang den tages på, kan den måske sidde på den med gult markerede muskel. Da musklerne er forbundet til hver sin funktionalitet i hånden, vil der komme to forskellige udfald i de opsamlede data. Der skal derfor tages højde for Myo'ens placering på armen, når der opsamles data.\\
Dette bliver nærmere beskrevet i sektion \ref{sec:optagelsedata}.

\section{Machine Learning}
\label{sec:machineLearning}
I følgende sektion vil den grundlæggende teori for brugen af machine learning blive belyst. Machine learning er læren om at få computere til at selv at "tænke" uden rent faktisk at programmere dem.

Formålet med machine learninger er, at lade en program lære af opsamlet data, i stedet for at programmere det. Der er flere typer af machine learning\citep{PatternBishop}:
\begin{description}
	\item[Supervised learning] er hvor træningsdata inputtet sammen med det ønskede output bliver givet til computeren. Her bliver den altså lært hvilket output et givent input passer til.
	\item[Unsuperviced learning] er derimod hvor der gives træningsdata, hvor der ikke er et tilhørende matchene output. Her er det altså op til computeren at finde et mønster, og dermed at kunne genkende lignede mønstre ud fra træningsdataet.
	\item[Reinforcement learning] er hvor computeren befinder sig i et dynamisk miljø og der skal findes løsninger for at komme nærmest målet. Reinforcement learning er bl.a. brugt i selvkørende biler.
\end{description}

I dette projekt er der arbejdet med Supervised learning. Der bliver optaget data fra Myo'en og fortalt hvad inputtet er, og dermed også hvad den skal give af output. Inputtet er et specifikt datasæt der bliver opsamlet. Der bruges derfor det der hedder \textit{inductive bias}. Den specifikke data skal bruges til at geneknde et mønster på noget nyt data. Det specifikke skal bruges til at genkende det generelle. Omvendt er \textit{deduction} at gå den anden vej, fra det generelle til det specifikke.


%generalisering
%inductive bias


%Træningssæt
\todola{Der bruges training set, så alle begreber er på engelsk}

\subsection{Proces}
\todola{Hvordan gøres det, beskriv trinene}
\subsection{Problematik}

%Kurve med validationsfejl kontra fejl/ fejl på træningssættet. 

\subsubsection{overfitting}
%Varians - sensitivitet over for individuelle data punktet i træningssæt, fejl i træningssættet
\subsubsection{underfitting}
%Bias - 

\subsection{Løsninger}
%test set/ validation set
%\\ Cross validation
%\\ Hold out
