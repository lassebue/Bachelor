\thispagestyle{fancy}
\chapter{Design og Implementering}
\label{chp:designogimpl}

Det er fra starten af projektet besluttet at applikationerne til systemet udvikles i Windows Presentation FOundation (WPF) i Microsoft Visual Studio, da dette er et framework der er arbejdet med tidligere, og det er derfor her der hurtigt ville kunne udviklet, da dette allerede er kendt.

\subsection{Data Collection Application}
\label{sec:datacoll}

\subsection{CrustCrawler Application}
\label{sec:ccapp}

Denne applikation stiller krav til anvendelsen af den opsamlede data og den model træning der er lavet. Applikationen skal være i stand til at åbne og lukke CrustCrawler-kloen ved hjælp af genkendelse af poses med Myo Armbånd. Denne applikation er også udviklet i WPF for at give den en brugergrænseflade, hvormed det er mulgit at se hvilken orientering Myo armbåndet har, og starte og stoppe recognizing af EMG signaler.

\subsubsection{Software arkitektur}
Det er valgt til CrustCrawler Applikationen at den skal udvikles i en Model-View-ViewModel (MVVM) arkitektur. MVVM er et WPF design pattern, der er brugt i mange applikationer med grafisk brugergrænseflade. Dette er et naturligt valg da der udvikles på en XAML platform. I MVVM arkitekturen er View'et separeret fra modellen, hvor funktionaliteten ligger. Der er begge koblet sammen af en ViewModel. Dette der udvikling af applikationen mere overskuelig og nem at gå til, og vedligeholdelse vil også være mere lige til. Desuden gør dette bl.a. at det er muligt at udvikle et nyt View uden at ændre i det gamle og bytte det ud når det ønskes.\\

\myFigure{MVVM}{MVVM Arkitektur}{fig:mvvm}{0.8}

På figur \ref{fig:mvvm} ses den overordnede struktur i hvordan klasserne er placeret. Denne figur er for at give et lille overblik, og der er derfor ikke taget alle klasser med.

\subsubsection{Brugergrænseflade}