\thispagestyle{fancy}
\chapter{Indledning}
\label{chp:indledning}
Målet med dette projekt er at undersøge mulighederne for at lave EMG-signalgenkendelse på musklerne i underarmen vha. machine learning. Herved vil håndbevægelser kunne genkendes af systemer, og anvendes som bruger input. Dette kan bruges for at gøre forskellige systemer og applikationer mere intuitive, og mere ligefremme at bruge.\\

I disse år er der stor interesse indenfor Virtual Reality. Her bliver der hele tiden udviklet nye løsninger til at interagere med de virtuelle miljøer. Inden for dette område vil genkendelse af håndbevægelser give brugere mulighed for at påvirke deres virtuelle omgivelser langt mere naturligt end ved brug af tastetur, mus eller controller.\\

I projektet vil der blive set på myoelektriske proteser og exoskeletter. Der vil blive undersøgt, hvordan disse fungerer, og om der kunne findes alternative styringsmetoder.
Kigger man på brugen af elektriske proteser og exoskeletter, vil genkendelse af håndbevægelser muligvis kunne gøre interfacet mellem bruger og maskine mere intuitivt.\\

Bevægelsesgenkendelse vil muligvis kunne anvendes til styring af robotarme. I dag er styringen af disse i mange tilfælde langt fra intuitiv. Derfor vil det i projektet blive undersøgt, hvorvidt genkendelse af håndbevægelser vil kunne anvendes som brugergrænseflader til sådanne robotarme. \\

Dette projekt betragtes som et undersøgelsesprojekt, hvor forskellige eksisterende løsninger undersøges. Herefter vil fordele og ulemper blive belyst. I projektet vil det blive undersøgt, om man vha. machine learning kan lave genkendelse af håndbevægelser ved brug af EMG-signaler fra underarmen, og endvidere anvende det til styring af en robotarm.

\vfill

\section{Læsevejledning}
\subsection*{Ordliste}

\begin{tabular}{p{90pt}p{240pt}}
	Pose & Håndbevægelse, der skal registreres eller bruges\\ \hline
	CrustCrawler & Robotarm med gribeklo - findes på ingeniørhøjskolen\\ \hline
	EMG & Elektromyografi - måling af elektrisk aktivitet i muskler.\\ \hline
	MATLAB COM Automation Server & Protokol, der lader en applikation styre en MATLAB server\citep{RefWorks:5}\\ \hline
	DCA & Data Collection Applicaton\\
\end{tabular}