\thispagestyle{fancy}
\chapter{Indledning}
\label{chp:indledning}
I Dette projekt arbejdes der med machine learning på data der er opsamlet på EMG signaler i underarmen. Dette kan bruge for at gøre forskellige systemer og applikationer mere intuitive, og mere ligefrem at bruge. Projektet er inspireret af elektriske proteser. Derfor er formålet med dette projekt at udvikle et system hvor der kan opsamles noget data ét sted af en testbruger, og et andet sted skal det være muligt at anvende denne opsamlede data, via machine learning. En CrustCrawler bliver brugt som en form for protese hvor der med en knyttet hånd ville kunne bruge CrustCrawleren til at gribe ud efter genstande. Det vil altså kunne lede til at man kan sætte en klo på enden af en arm der ingen hånd har. Her vil personen kunne bruge kloen selvom han ingen hånd har, da der stadig sendes elektriske impulser ud til underarmen, selvom håndet ikke eksisterer.

I projektet vil der blive lavet én anvendelsesmulighed, og meningen er at dette skal åbne op for at vise at der er masser af muligheder for anvendelse af EMG signaler. Dette kan både være i den vituelle værden, som for eksempel i Virtual reality, men også i den virkelige verden, hvor det kan bruges til styring af forskellige ting. Bl.a. er der meget fokus på udvikling af exo skelet på Hamme Neurocenter. Exo-skelettet stopper dog ved armen, og her ville en udvidelse være mulig, ved at få en hånd med på skelettet.

\section{Læsevejledning}
\subsection*{Ordliste}
\begin{tabular}{lll}
	Pose && Håndbevægelse der skal registreres eller bruges\\
	CrustCrawler && Robotarm med gribeklo som findes på ingeniørhøjskolen\\
	EMG && Elektromyografi - Måling af elektrisk aktivitet i muskler.
\end{tabular}