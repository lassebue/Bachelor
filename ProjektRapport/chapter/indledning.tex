\thispagestyle{fancy}
\chapter{Indledning}
\label{chp:indledning}
I Dette projekt arbejdes der med machine learning på data der er opsamlet på EMG signaler i underarmen. Dette kan bruge for at gøre forskellige systemer og applikationer mere intuitive, og mere ligefrem at bruge. Projektet er inspireret af elektriske proteser. Derfor er formålet med dette projekt at undersøge mulighederne for at lave EMG signalgenkendelse på musklerne i underarmen og bruge det til udvikling af et system der kan opsamle data ét sted af en testbruger, for derefter at være muligt at anvende den opsamlede data et andet sted. Dette vil i sidste ende kunne være en protese eller en exoskelethånd.\\

Kigger man på Problemstillingen omkring brug af elektriske proteser og exoskeletter, er der flere forsøg under udvikling og forskellige måder andre har prøvet på at finde en løsning til problemet. I projektet vil nogle af disse løsninger blive belyst, og der vil blive udviklet en tredje løsning, som vil blive holdt op imod de andre løsninger der er udviklet.

I projektet bliver en CrustCrawler brugt som en form for protese, hvor der med en knyttet hånd ville kunne bruge CrustCrawleren til at gribe ud efter genstande. Det vil altså kunne lede til at man kan sætte en klo på enden af en arm der ingen hånd har. Her vil personen kunne bruge kloen selvom han ingen hånd har, da der stadig sendes elektriske impulser ud til underarmen, selvom hånden ikke eksisterer.

Én anvendelsesmulighed vil blive udviklet i projektet, og meningen er at dette skal åbne op for at vise at der er masser af muligheder for anvendelse af EMG signaler.
\section{Læsevejledning}
\subsection*{Ordliste}
\todoall{Definer exoskelet kort}
\begin{tabular}{lll}
	Pose && Håndbevægelse der skal registreres eller bruges\\
	CrustCrawler && Robotarm med gribeklo som findes på ingeniørhøjskolen\\
	EMG && Elektromyografi - Måling af elektrisk aktivitet i muskler.\\
	Exoskelet && ??
\end{tabular}