\thispagestyle{fancy}
\chapter{Indledning}
\label{chp:indledning}
Målet med dette projekt er at undersøge mulighederne for at lave EMG signalgenkendelse på musklerne i underarmen vha. machine learning. Herved vil håndbevægelser kunne genkendes af systemer, og anvendes bruger input. Dette kan bruges for at gøre forskellige systemer og applikationer mere intuitive, og mere ligefrem at bruge.\\

I disse år er der stor interesse indenfor virtual reality, her bliver der hele tiden udviklet nye løsninger til at interagerer med de virtuelle miljøer. Inden for dette område vil genkendelse af håndbevægelser giver brugere mulighed for at påvirke deres virtuelle omgivelser langt mere naturligt end ved brug af tastetur, mus eller controller.\\

I projektet vil der blive set på myoelektriske proteser og exoskeletter. Der vil blive undersøgt hvordan disse fungerer og om der kunne findes alternative styringsmetoder.
Kigger man på brugen af elektriske proteser og exoskeletter, vil genkendelse af håndbevægelse muligvis kunne gøre interfacet mellem brugeren og maskine \todoall{find andet ord} mere brugervenlig.\\

Bevægelsesgenkendelse vil muligvis kunne anvendes til styring af robotarme. I dag er styringen af disse, i mange tilfælde langt fra intuitiv. Derfor vil det i projektet blive undersøgt, hvorvidt genkendelse af håndbevægelser vil kunne anvendes som brugergrænseflader til sådanne robotarme. \\

Dette projekt betragtes som et forskningsprojekt\todoall{evt. andet ord en "forskning"}, hvor forskellige eksisterende løsninger undersøges, herefter vil fordele og ulemper blive belyst. I projektet vil det blive undersøgt om man vha. machine learning kan lave genkendelse af håndbevægelser ved brug af EMG signaler fra underarmen, og videre anvende det til styring af proteser og exoskelet.

\todoall{Slet når færdig}
%Det vil i projektet forsøges at lave 
%I projektet vil nogle af disse løsninger blive belyst, og der vil blive udviklet en tredje løsning, som vil blive holdt op imod de andre løsninger der er udviklet.
%
%
%Disse systemer kan være inden for medicinalindustrien, 
%Projektet er inspireret af elektriske proteser og exoskeletter.\\
%
%I projektet bliver en CrustCrawler brugt som en form for protese, hvor der med en knyttet hånd ville kunne bruge CrustCrawleren til at gribe ud efter genstande. Det vil altså kunne lede til at man kan sætte en klo på enden af en arm der ingen hånd har. Her vil personen kunne bruge kloen selvom han ingen hånd har, da der stadig sendes elektriske impulser ud til underarmen, selvom hånden ikke eksisterer.
%
%Én anvendelsesmulighed vil blive udviklet i projektet, og meningen er at dette skal åbne op for at vise at der er masser af muligheder for anvendelse af EMG signaler.
\section{Læsevejledning}
\subsection*{Ordliste}
\todoall{Definer kort}
\begin{tabular}{p{80pt}lp{240pt}}
	Pose && Håndbevægelse der skal registreres eller bruges\\
	CrustCrawler && Robotarm med gribeklo som findes på ingeniørhøjskolen\\
	EMG && Elektromyografi - Måling af elektrisk aktivitet i muskler.\\
	Exoskelet && ??\\
	Myo && ??\\
	MATLAB Automation Server && ??\\
\end{tabular}