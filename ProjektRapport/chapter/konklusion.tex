\thispagestyle{fancy}
\chapter{Konklusion}
\label{chp:konklusion}

Formålet med dette projekt har været at belyse muligheden for anvendelse af EMG-genkendelse og machine learning til styring af en robot arm. Ud fra resultaterne i kapitel \ref{chp:resultaterogdisk} kan det konkluderes, at dette er lykkedes.\\\\
Undersøgelsen der igennem projektet er udført, resulterede i tre overordnede dele: Data Collection Application, Modeltræning og Testapplikation.\\\\
Data Collection Application resulterede i et værktøj til opsamling og upload af data optaget med Myo’en. De opsamlede data gemmes i .csv-filer, som kan importeres af databehandlingsværktøjer.\\\\
Modeltræning har resulteret i, at der er blevet trænet machine learning classifiers, ud fra data opsamlet med Myo’en. Det konkluderes, at Bagged Trees modellen i MATLAB’s Classification Learner er den mest effektive learner til at træne modeller med høj nøjagtighed.\\\\
Testapplikation har vist, at trænede machine learning modeller kan anvendes, til posegenkendelse og intuitiv styring af en CrustCrawler-robotarm. Endvidere konkluderes det, at testapplikationen ved videreudvikling ville kunne agere brugergrænseflade til andre systemer med behov for intuitiv styring.\\\\
Det er den overordnede konklusion, at projektet er vellykket.\\\\
Det er bevist, at det ved brug af EMG-genkendelse og machine learning, er muligt at lave et intuitivt brugerinterface til en CrustCrawler-robotarm. 

\label{SidsteSide}