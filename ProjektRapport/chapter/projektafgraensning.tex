\thispagestyle{fancy}
\chapter{Projektafgrænsning}
\label{chp:projektafgraensning}
I dette afsnit gennemgås en specificering af projektets omfang. %Det er ud fra problemformuleringen i kapitel \ref{chp:problemformulering}, blevet vurderet hvad 
%\todoall{Her skal skrives mere når Problemformulering er lavet}
\section{Projekt omfang}
Det ses ud fra problemformuleringen, at flere systemer vil kunne optimeres og gøres mere intuitive med genkendelse af poses vha. machine learning i stedet for brug af knapper, mobilapplikationer og joystik. \\

I forbindelse med armproteser, hvor der allerede findes en løsning, hvor proteser kan styres vha. EMG-signaler, er problemet, at der anvendes kommandoer, som brugerne skal huske. Endvidere er der kun få mulige kommandoer, således er brugerne ofte er nød til at skifte protese-bevægelser, bestemt af kommandoerne. Desuden gør kommandoerne, at brugerne ikke kan lave intuitive muskelbevægelser for de poses, de gerne vil lave med protesen. 

Her vil det gennem projektet bliver undersøgt, om det er muligt at udvikle pose genkendelse vha. machine learning, da man, hvis det er muligt, vil kunne lave intuitiv styring af flere protese-bevægelser. \\
Som tidligere nævnt kan en Myo opsamle EMG-signaler fra armen og bruges til pose-genkendelse. Denne funktionalitet ville kunne anvendes i denne sammenhæng. Dog har Myo'en en stor begrænsning i forhold til antallet af poses, der kan genkendes. Denne løsning vil således ikke være meget bedre end den eksisterende.\\

Derfor vil der i dette projekt blive udviklet et program til genkendelse af EMG-signaler vha. machine learning. \\
Ved machine learning, har man behov for relativt store mængder data. Har man ingen data, kan man ikke lave machine learning. I den forbindelse ville det være hensigtsmæssigt at have en datasamling. Til at indsamle disse data har Myo'en en anden funktion, der gør det muligt at opsamle de rå EMG-data. Således vil det være muligt anskaffe signal-data fra underarmen, som vil kunne anvendes til træning af machine learnings klassifikationsmodeller. \\
I projektet er det intentionen at anvende de trænede modeller til en konkret anvendelse. \\\\
Det færdige projekt skal således indeholde:
\begin{myItemize}
\item Et system, der skal kunne:
	\begin{myItemize}
		\item Opsamle EMG-data fra underarmen.
		\item Uploade data til en samlet datasamling.
		\item Bruge den indsamlede data til træning af en genkendelsesalgoritme.
		\item Genkende forskellige poses i realtid vha. en udviklet genkendelsesalgoritme.
		\item Lave relevant anvendelse af algoritmen.
	\end{myItemize}
%\item Undersøgelse af eksisterende projekter indenfor machine learning.
%\item Undersøgelse af det udviklede system.
\end{myItemize}
