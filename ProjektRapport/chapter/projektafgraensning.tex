\thispagestyle{fancy}
\chapter{Projektafgrænsning}
\label{chp:projektafgraensning}
I dette afsnit gennemgås opsatte læringsmål samt en specificering af projektets omfang. 
\section{Læringsmål}
Da dette er en bachelor-opgave på Ingeniørhøjskolen ved Aarhus Universitet, er der opsat en række læringsmål, der skal opfyldes igennem projektet. I det følgende er de forskellige læringsmål gennemgået, og der beskrives, hvordan projektet opfylder læringsmålene.
\begin{myItemize}
	\item Omsætte forskningsresultater samt naturvidenskabelig og teknisk viden til anvendelse ved udviklingsopgaver og ved løsning af teknologiske problemstillinger.
	\begin{myItemize}
		\item Undersøgelse af teori bag machine learning og projekter med EMG anvendelse, hvorefter relevante erfaringer integreres i dette projekt. \todoall {Hvad kan der lige skrives her}
	\end{myItemize}
	\item Søge, analysere og vurdere ny viden indenfor relevante områder.
	\begin{myItemize}
		\item Der søges ny viden i projekter med proteser og exoskeletter. 
	\end{myItemize}
	\item Udvikle nye løsninger.
	\begin{myItemize}
		\item Dette gøres ved at lave en løsning, hvor EMG-signaler bruges sammen med machine learning, hvor systemet kan genkende håndbevægelser, og en robotarm vil reagere herpå.
	\end{myItemize}
	\item Anvende ingeniørfaglig teori og metode på en systematisk måde.
	\begin{myItemize}
		\item Projektet gennemføres vha. diverse specifikke principper og modeller, der undervejs giver en god projektgennemførelse. Dette er bl.a. udarbejdelse af kravspecifikation.
	\end{myItemize}
	\item Vurdere og forklare projektresultater for ingeniører og andre målgrupper, skriftlig og mundtligt.
	\begin{myItemize}
		\item Dette læringsmål bliver udfyldt med denne rapport og ved den efterfølgende mundtlige eksamen.
	\end{myItemize}
	\item Reflektere over anvendelsen af projektresultaterne i relation til sociale, organisatoriske, miljømæssige, arbejdsmiljømæssige, økonomiske og etiske konsekvenser.
	\begin{myItemize}
		\item I projektet bliver der reflekteret igennem kapitlerne \ref{chp:testapp}, \ref{chp:resultaterogdisk} og \ref{chp:fremtidigtarbejde} over, hvordan hele forløbet er gået, og hvilke konsekvenser det har eller kan medføre.
	\end{myItemize}
\end{myItemize}

De opstillede læringsmål vil blive brugt igennem projektet ved undersøgelse af eksisterende projekter og udviklingen af gruppens eget system.\\


\todoall{Her skal skrives mere når Problemformulering er lavet}
\section{Projekt omfang}
Det ses ud fra problemformuleringen, at flere systemer vil kunne optimeres og gøres mere intuitive med genkendelse af poses vha. machine learning i stedet for brug af knapper, mobilapplikationer og joystik. \\

I forbindelse med armproteser, hvor der allerede findes en løsning, hvor proteser kan styres vha. EMG-signaler, er problemet, at der anvendes kommandoer, som brugerne skal huske. endvidere er der kun få mulige kommandoer, således er brugerne ofte nød til at skifte protese-bevægelser, bestemt af kommandoerne. Desuden gør kommandoerne, at brugerne ikke kan lave intuitive muskelbevægelser for de poses, de gerne vil lave med protesen. 

Her vil det gennem projektet bliver undersøgt, om det er muligt at udvikle pose genkendelse vha. machine learning, da man, hvis det er muligt, vil kunne lave intuitiv styring af flere protese-bevægelser. \\
Som tidligere nævnt kan en Myo opsamle EMG signaler fra armen og bruges til pose-genkendelse. Denne funktionalitet ville kunne anvendes i denne sammenhæng. Dog har Myo'en en stor begrænsning i forhold til antallet af poses, der kan genkendes. Denne løsning vil således ikke være meget bedre end den eksisterende.\\

Derfor vil der i dette projekt blive udviklet et program til genkendelse af EMG-signaler vha. machine learning. \\
Ved machine learning, har man behov for relativt store mængder data. Har man ingen data, kan man ikke lave machine learning. I den forbindelse ville det være hensigtsmæssigt at have en datasamling. Til at indsamle disse data har Myo'en en anden funktion, der gør det muligt at opsamle de rå EMG-data. Således vil det være muligt anskaffe signal-data fra underarmen, som vil kunne anvendes til træning af machine learnings klassifikationsmodeller. \\
I projektet er det intentionen at anvende de trænede modeller til en konkret anvendelse. \\\\
Det færdige projekt skal således indeholde:
\begin{myItemize}
\item Et system, der skal kunne:
	\begin{myItemize}
		\item Opsamle EMG-data fra underarmen.
		\item Uploade data til en samlet datasamling.
		\item Bruge den indsamlede data til træning af en genkendelsesalgoritme.
		\item Genkende forskellige poses i realtid vha. en udviklet genkendelsesalgoritme.
		\item Lave relevant anvendelse af algoritmen.
	\end{myItemize}
\item Undersøgelse af eksisterende projekter indenfor machine learning.
\item Undersøgelse af det udviklede system.
\end{myItemize}
