\thispagestyle{fancy}
\chapter{Projktafgrænsning}
\label{chp:projektafgraensning}
Da dette er en bachelor opgave på Ingeniørhøjskolen ved Aarhus Universitet, er der opsat en række læringsmål der skal opfyldes igennem projektet. I det følgende der de forskellige læringsmål gennemgået og en beskrivelse af hvordan projektet opfylder læringsmålene.
\begin{itemize}
	\item Omsætte forskningsresultater samt naturvidenskabelig og teknisk viden til anvendelse ved udviklingsopgaver og ved løsning af teknologiske problemstillinger.
	\begin{itemize}
		\item 
	\end{itemize}
	\item Søge, analysere og vurdere ny viden indenfor relevante områder.
	\begin{itemize}
		\item Der er søges ny viden i projekter med proteser og exoskeletter. 
	\end{itemize}
	\item Udvikle nye løsninger.
	\begin{itemize}
		\item Dette er gøres ved at lave en løsning hvor EMG signaler bruges sammen med machine learning, hvor systemet kan genkende håndbevægelser og en robotarm vil reagere herpå.
	\end{itemize}
	\item Anvende ingeniørfaglig teori og metode på en systematisk måde.
	\begin{itemize}
		\item Projektet gennemføres vha. diverse specifikke principper og modeller, der undervejs giver en god projektgennemførsel. Dette er ting som bl.a. udarbejdelse af kravspecifikation.
	\end{itemize}
	\item Vurdere og forklare projektresultater for ingeniører og andre målgrupper, skriftlig og mundtligt.
	\begin{itemize}
		\item Dette læringsmål bliver udfyldt med denne rapport og ved den efterfølgende mundtlige eksamen.
	\end{itemize}
	\item Reflektere over anvendelsen af projektresultaterne i relation til sociale, organisatoriske, miljømæssige, arbejdsmiljømæssige, økonomiske og etiske konsekvenser.
	\begin{itemize}
		\item I projektet bliver der reflekteret igennem kapitlerne \ref{chp:designogimpl}, \ref{chp:resultaterogdisk} og \ref{chp:fremtidigtarbejde} over hvordan hele forløbet er gået og hvilke konsekvenser det har eller kan medføre.
	\end{itemize}
\end{itemize}

De opstillede læringsmål vil blive brugt igennem projektet ved undersøgelse af eksisterende projekter og udviklingen af gruppens eget system.

\todoall{Her skal skrives mere når Problemformulering er lavet}