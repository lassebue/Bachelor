\thispagestyle{fancy}
\vfill
\section*{Resumé}
\label{sec:resume}
Denne rapport beskriver bachelorprojekt nr. 15171, EMG Recognition Using Machine Learning på 7. semester på ingeniørhøjskolen i Aarhus, Informations- og Kommunikationsteknologi ved Aarhus Universitet. Projektets problemstilling omhandler undersøgelse af mulighed for EMG-genkendelse vha. machine learning til brug som intuitiv brugergrænseflade til anvendelse på en robot-arm. I projektet er der udviklet et dataopsamlingsværktøj til at optage muskelaktivitet i underarmen. De opsamlede data kan gemmes online, hvor alle kan anvende den. Dataene bliver trænet med machine learning modeller til at genkende håndbevægelser, som f.eks. at knytte hånden. Det er lykkedes at få modellerne til at genkende visse håndbevægelser. Der er derfor udviklet en testapplikation for at vise, at machine learning modeltræningen virker. Testapplikationen genkender visse håndbevægelser og en robot-arm vil reagere herpå ved at gribe eller slippe med dens klo. Det er muligt at udvide applikationen med flere håndbevægelser hvis dette ønskes med henblik på nye anvendelsesmuligheder.

\vfill

\section*{Abstract}
\label{sec:abstract}
This thesis describes the bachelor's final project no. 15171, EMG Recognition Using Machine Learning on the 7th semester at the Aarhus School of Engineering, Information Technology at Aarhus University. The thesis includes research in the posibility of doing EMG recognition using machine learning for an intuitive user interface. It was the intension to use the trained models to develop a test app for use on a robotic arm. There is developed a data collection tool, for collecting muscle activity, in the forearm. The data collection can be saved online, for collaborate use. The data will be trained with machine learning models to recognize hand gestures. A testapplication is developed to show, that the machine learning modeltraining works. The testapplication recognizes hand gestures, and a robotic-arm reacts by closing or opening the claw. It is possible to extend the application with more hand gestures for new application possibilities.
\vfill