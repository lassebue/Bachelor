Når import og feature extraction er udført, er den næste del træningen af en machine learningmodellerne. Udførelsen af dette vil i denne sektion være beskrevet.

For at finde ud af om det er muligt at genkende poses med Myo'en og machine larning, er der først trænet modeller , hvor dataet er opsamlet fra Myo'en hvor den sidder på én måde. Det blev bekræftet at det var muligt at genkende poses vha. machine learning på EMG-signaler fra Myo'en. Efterfølgende er nye data indsamlet hvor der mellem hver indsamling er lavet en lille rotation af Myo'en på armen, og der opsamles dermed data i positioner hele vejen rundt på underarmen.

Træningen af machine learning modellerne foregår i Matlabs Classification Learner\citep{matlabClassificationLearner}. Classification Learner er et værktøj der giver mulighed for træning af mange forskellige slags machine learning modeller. Under træningen er flere forskellige afprøvet for at se deres nøjagtighed. Et par af disse er gennemgået i kapitel \ref{chp:mlm}.

\myWrapFigure{matlabModeller}{De forskellige machine learning modeller man kan træne sit data ud fra. Der bliver valgt Bagged Trees da den giver en precis model. Dette er der beskrevet mere om i sektion \ref{sec:baggedTrees} }{fig:matlabModeller}{0.4}{l}

Classification Learneren åbnes og der startes en ny session. Her skal valideringsmetoden vælges. Holdout-validation vælges da det er en stor mængde data der skal trænes på. Med Holdout-validering vælges en procentdel af dataet der holdes ude af træningen. Den del der holdes ude bruges til test for at se hvor god modellen er\citep{matlabValidation}.

Træningen er modellerne er nu klar til at blive gennemført og der skal vælges hvilken machine learning model der skal bruges. På figur \ref{fig:matlabModeller} ses valgmulighederne.
Bagged Trees vælges, og modellen trænes.
Der kan trænes flere modeller i én session, det bliver derfor nemt at sammenligne dem, for at se hvilke der er bedst. Når en model er trænet, vises den i historikken i Classification Learneren, hvor præcisionen er vist.












