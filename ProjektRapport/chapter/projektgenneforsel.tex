\thispagestyle{fancy}
\chapter{Projektgennemførsel}
\label{chp:projektgennemforsel}

Gennemførslen af projektet har forløbet lidt anderledes end tidligere semesterprojekter. Dette er grundet at projektet er formet som et undersøgelsesprojekt mere end et udviklingsprojekt. I dette kapitel er der beskrevet nogle metoder og værktøjer der er blevet benyttet igennem projektet for at få god struktur og

\section{Projektstyring}
\label{sec:projektstyring}

Til styring af projektet gennem det seneste semester, er der brugt dele fra metoden SCRUM.
Én af værktøjerne fra scrum der er anvendt, er scrumboardet. Her er alle opgaverne nedbrudt i mindst mulig størrelse, og sat op på et scrumboard. På den måde kan gruppen nemt overskue hvilke opgaver der er lavet, hvilke der er under udvikling og hvilke der mangler at blive lavet. På figur \ref{fig:scrum} ses et eksempel på boardet fra en dag i udviklingsprocessen

\myFigure{SCRUM.JPG}{Scrumboard - Placeholder}{fig:scrum}{0.5}
\todoall{evt. pænere billede}

\subsection{Vejledermøder}
Der er i løbet af projektet holdt ugentlige vejledermøder, med undtagelse af nogle uger. Referater herfra kan ses i bilag \ref{bilag:vejl}. Møderne som som regel ligget torsdag kl. 13, og har været brugt til at få opklarende spørgsmål besvaret og der er ud fra vejledermøderne taget diverse beslutninger igennem projektforløbet.

\subsection{Tidsplan}
\label{sec:tidsplan}
Der blev ved projektstart udarbejdet en tidsplan for at have nogle holdepunkter genne projektet. Igennem forløbet er der lavet justeringer i tidsplanen og den endelige tidsplan ses på figur \ref{fig:tidsplan}

\myFigure{tidsplan}{Tidsplan for projektgennemførslen fra og med september til og med december}{fig:tidsplan}{1}

\section{Udviklingsværktøjer}
\label{sec:Udviklingsvaerktojer}

\subsection*{Microsoft Visual studio 2013}
Udviklingen af applikationerne Data Collection Application og CrustCrawler Application er sket i WPF i MS Visual Studio 2013. Dette er et stærkt værktøj til udvikling af GUI programmer, og er valgt ud fra tidligere erfaringer med værktøjet.

\subsection*{Mathworks Matlab R2015b}
Matlab er i projektet blevet brugt flere forskellige steder. Under eksekveringen af systemet, er det også i spil flere gange, bl.a. er det matlab funktioner der sender kommandoer ud til CrustCrawleren.

\subsection*{Microsoft Visio 2016}

Til udvikling af software er diverse diagrammer lavet i MS Visio 2016, da dette tidligere er brugt, og er derfor kendt af gruppen. Heri laves Use Case- og klassediagrammer m.fl.

\subsection*{TexStudio}
Dokumentationen i dette bachelorprojekt inklusiv denne rapport er skrevet i \LaTeX, i TexStudio. Dette er gjort da det er nemt at arbejde med når der lave længere rapporter.

\subsection*{SmartGit/GitHub}
SmartGit clienten er brug til versionsstyring af projektets software, hvor repository har ligget hos GitHub.com

\subsection*{Parse.com}
Parse.com \citep{RefWorks:11} er brugt som database til alle de opsamlede data fra Myo'en.

