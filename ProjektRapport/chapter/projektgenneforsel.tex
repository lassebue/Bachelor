\thispagestyle{fancy}
\chapter{Projektgennemførelse}
\label{chp:projektgennemforsel}

Gennemførelsen af projektet er forløbet lidt anderledes end tidligere semesterprojekter. Dette er grundet i, at projektet er formet som et undersøgelsesprojekt mere end et udviklingsprojekt. I dette kapitel er der beskrevet nogle metoder og værktøjer, der er blevet benyttet igennem projektet for at få god struktur igennem forløbet.

\section{Projektstyring}
\label{sec:projektstyring}

Til styring af projektet gennem det seneste semester, er der brugt dele fra metoden SCRUM.
Ét af de værktøjer fra SCRUM, der er anvendt, er scrumboardet. Her er alle opgaverne nedbrudt til mindst mulig størrelse, og sat op på et scrumboard. På den måde kan gruppen nemt overskue, hvilke opgaver der er lavet, hvilke der er under udvikling, og hvilke der mangler at blive lavet. På figur \ref{fig:scrum} ses et eksempel på boardet fra en dag i udviklingsprocessen. 

\myFigure{SCRUM.JPG}{Todo, en kolonne med de opgaver der skal laves. In progress, kolonne med de igangværende opgaver, her skal være så få opgaver som muligt på samme tid. Testkolonnen, tager den person der ikke har lavet opgaven og tester om det virker, eller reviewer noget arbejde. Til sidst er done kolonnen hvor en opgave flyttes over når den er lavet og testet.}{fig:scrum}{0.5}
\todokr{evt. pænere billede}

Hver dag når gruppen samledes i bachelorlokalet blev der hold et daily scrum møde, hvor hver især blev sat ind i hvad hver havde lavet dagen forinden, havd de skulle lave i dag og om der var noget i vejen for at de kunne arbejde videre. Dette er kun et kort møde på 5-10 minutter, for at have styr på hvor langt udviklingen var og om der var en der sad fast i en opgave.

\subsection{Vejledermøder}
Der er i løbet af projektet holdt ugentlige vejledermøder, med undtagelse af få uger. Referater fra disse møder kan ses i bilag \ref{bilag:vejl}. Møderne har som regel ligget torsdag kl. 13, og har været brugt til at få opklarende spørgsmål besvaret. Ud fra vejledermøderne er der taget diverse beslutninger igennem projektforløbet.

\subsection{Research metode}
Under udvikling er der brugt en research model, der viser at det først er blevet forsøgt om det kan lade sig gøre at lave en machine learnings model der virker. Gør den det forsøges det at implementere systemet med orientering. På figur \ref{fig:rm} ses hvordan det er forløbet.

\myFigure{ResearchMetode}{Det ses hvordan udvikling og modeltræning er gennemløbet to gange. Først undersøges om det kan lade sig gøre at udvikle et dataindsamlingsværktøj der kan give data, hvorpå en model kan trænes. Derefter udvikles en testapplikation, for at se om poses kunne genkendes. Var dataene ikke gode nok, blev der opsamlet nye data. Da det blev bekræftet at det virkede, begyndte videreudvikling på Dataopsamlingsprogrammet, således at orienteringen også blev opsamlet og brugt. Ny opsamling blev foretaget, og ny model trænet. Testapplikationen blev videreudviklet og der blev igen testet om det virkede.}{fig:rm}{0.7}

På tidsplanen figur \ref{fig:tidsplan} ses hvordan der er arbejdet på dataopsamlingsprogram, opsamle data og machine learning modeltræningen over to gange. Igennem rapporten er det beskrevet efter orienteringen er taget med.


\subsection{Tidsplan}
\label{sec:tidsplan}
Der blev ved projektstart udarbejdet en tidsplan for at have nogle holdepunkter igennem projektet. Igennem forløbet er der lavet justeringer i tidsplanen, og den endelige tidsplan ses på figur \ref{fig:tidsplan}

\myFigure{tidsplan}{Tidsplan for projektgennemførelsen fra og med september til og med december}{fig:tidsplan}{1}

\section{Udviklingsværktøjer}
\label{sec:Udviklingsvaerktojer}

\subsection*{Microsoft Visual Studio 2013}
Udviklingen af applikationerne Data Collection Application og CrustCrawler Application er sket i WPF i MS Visual Studio 2013. Dette er et stærkt værktøj til udvikling af GUI programmer, og er valgt ud fra tidligere erfaringer med værktøjet.

\subsection*{Mathworks Matlab R2015b}
Matlab er i projektet blevet brugt flere forskellige steder. Under eksekveringen af systemet, er det også i spil flere gange, bl.a. er det matlab-funktioner der sender kommandoer ud til CrustCrawleren.

\subsection*{Microsoft Visio 2016}

Til udvikling af software er diverse diagrammer lavet i MS Visio 2016, da dette tidligere er brugt og derfor er kendt af gruppen. Heri laves Use Case- og klassediagrammer m.fl.

\subsection*{TexStudio}
Dokumentationen i dette bachelorprojekt inklusiv denne rapport er skrevet i \LaTeX, i TexStudio. Dette er gjort, da det er nemt at arbejde med, når der laves længere rapporter.

\subsection*{SmartGit/GitHub}
SmartGit klienten\citep{smartgit} er brugt til versionsstyring af projektets software, hvor repository har ligget hos GitHub.com\citep{github}

\subsection*{Parse.com}
Parse.com \citep{RefWorks:11} er brugt som database til opbevaring af .csv filer. På Parse er det nemt at administrere, og udtrække filerne igen.