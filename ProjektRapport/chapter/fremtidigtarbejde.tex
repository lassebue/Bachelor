\thispagestyle{fancy}
\chapter{Fremtidigt arbejde}
\label{chp:fremtidigtarbejde}
Systemet giver mulighed for mange udvidelser og anvendelsesmuligheder. Nogle af de muligheder der ses som oplagte at implemetere fremadrettet er i dette kapitel beskrevet.

\section{Machine learning model og poses}
Fremadrettet vil der blive udviklet flere modeller, hvori der indgår nye poses. Disse ville kunne blive afprøvet mod hinanden for at finde de bedste og anvende dem. Opsamling af data kan også ske fra mange forskellige personer og det vil dermed være muligt at lave modeller der er baseret på mange personer i stedet for de kun er lavet på én person, hvilken skulle gøre modellen mere bred og dermed virke mere præcist på mange forskellige personer.

\section{Testapplikation}
Da testapplikationen er opbygget efter SOLID principperne\citep{RefWorks:10}, er det nemt at udvide den med mere funktionalitet, og flere poses. Nye poses kan hurtigt implementeres, og der kan laves en anvendelse hertil på CrustCrawleren. Roteres hånden ville man kunne lave en rotation af CrustCrawlerens klo, dette ville selvfølgelig gøre kloen mere mobil og anvendelig. En anden udvidelse der kunne være spændende at implementere, er brugen af Myo'ens orientering og accelerometer. Det ville være smart hvis CrustCrawleren kunne bevæge sig efter de bevægelser som Myo'en opfanger. Tippes Myo'en fremover, ville CrustCrawleren også tippes fremover. med denne funktionalitet ville robotarmen kunne bruges i stedet for brugerens egen arm. Dette ville være hensigtsmæssig hvis der skal arbejdes med hænderne i et miljø der ikke er sundt for de menneskelige lemmer.

Det vil også være aktuelt at se på at få CrustCrawleren til at reagere i realtid. Dette vil gøre anvendelsen meget mere intuitiv, da CrustCrawleren reagerer med det samme.

\section{Andre anvendelsesmuligheder}
testapplikationen der er udviklet er for at vise hvordan systemet kan anvendes. Dette er altså kun én af mange anvendelsesmuligheder af systemet. I det følgende er der beskrevet andre anvendelsesmuligheder.

\subsection{Exo-skelet}
Som tidligere nævnt, er der i et tidligere bachelorprojekt\citep{RefWorks:7} udviklet et exo-skelet til bevægelse af albueledet. Dette er med hensigt på genoptræning af patienter på Hammen Neurocenter. Med dette system ville man igen kunne udvikle et exo-skelet. Denne gang med fokus på genoptræning af hånden. Selvom en patient ikke er i stand til fx. at  åbne og lukke hånden, eller holde tommelfingeren oppe, vil der stadig blive sendt signaler fra hans hjerne til musklerne, når han forsøger at bevæge hånden. Med et exo-skelet på hånden vil det være muligt at genoptræne disse patienters føring i hånden og give dem håndens funktionalitet tilbage. Til dette skal der samles data op på flere poses, sådan at der er data til alle de poses, det ønskes at exo-skelettet skal hjælpe patienten med. Machine Learning modeller kan derefter trænes på dataen, og der vil nu kunne genkendes nye poses. Hvis der bruges bynamixel servoer vil implementeringen kunne foregå som det er blevet med CrustCrawleren, men der kan også anvendes andre muligheder for motorer og styring.

\subsection{Proteser}
Kigger man igen i samme retning, og vil hjælpe patienter, kan systemet give en person der har mistet sin hånd en meget lignende funktionalitet som en hånd ville have givet ham. Ved hjælp af en elektrisk protese ville det være muligt at anvende de trænede modeller til at kontrollere en robothånd. Da der er udviklet posegenkendelse ville det være en meget intuitiv måde at styre protesen på. Ligesom det beskrevne exo-skelet kan der træne flere modeller med nye poses i, der kan anvendes. Protesen vil igen på samme måde kunne implementeres som exo-skelettet, forskellen er kun, at det er en protese i stedet for et exo-skelet.

\subsection{Virtual Reality}
I Virtual Reality(VR) verdenen ville interaktionen kunne ændres og det vil være muligt at kunne anvende sine hænder i det virtuelle miljø i stedet for mus eller tastatur. I VR kan en virkelighed simuleres. Her vil brugen har arme og hænder være spændene at arbejde med, da de ville kunne genskabes virtuelt, så de er mulige at se i simuleringen. Der skal arbejdes med 3D grafik og både Myo'ens EMG sensorer og orientering ville skulle i brug for at simulere arme og hænder.














