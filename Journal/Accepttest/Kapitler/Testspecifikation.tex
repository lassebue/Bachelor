\chapter{Testspecifikation}
 \label{chp:testspecifikation}
 

\section{Testopstilling}

\subsection{Enheder som anvendes i testen:}

\begin{itemize}
\item Brugers enhed defineres som en Nokia Lumia 925 kørende Windows Phone 8.
\item Festivalsgæsts enhed defineres som en Nokia Lumia 520 kørende Windows Phone 8.
\end{itemize}


Nedenfor ses den fysiske testopstilling bestående af en Nokia Lumia 925 og en Nokia Lumia 520
\figur {spil2.jpg} {Testopstilling}{Fig:testopstilling}{0.68} 


\subsection{Opsatte testbrugere:}
Følgende bruger er oprettet på Facebook ved testent begyndelse. Igennem forløbet ville disse Facebook brugere benyttes.
\begin{itemize}
\item Bruger benytter Facebookprofilnavnet: Jens Testrup
\begin{itemize}
\item Brugernavn: hauswow@gmail.com
\item Adgangskode: semesterprojekt
\end{itemize}
\item Festivalgæst benytter Facebookprofilnavnet Ulla Testrup
\begin{itemize}
\item Brugernavn: mingleprojekt@gmail.com
\item Adgangskode: semesterprojekt
\end{itemize}
\end{itemize}

\subsection{Nulstilling af point}
Begge testbrugers optjente point nulstilles ved testens start i databasen.


\section{Godkendelseskriterier}
Acceptesten er afsluttet, når samtlige test cases er gennemført og godkendt.

Hvis fejl opstår under acceptest som umuliggør fortsættelsen af acceptesten afbrydes denne.

Hvis der opstår fejl i de enkelte test cases, men acceptesten fortsat kan fortsætte, så underkendes den enkelte test case. Accepttesten fortsættes herefter.




