\RequirePackage{etex}

% Vi benytter 2 sidet, a4 størrelse og dansk sprog. 
\documentclass[11pt,a4paper,twoside,danish]{memoir}


\usepackage[paper=A4,pagesize]{typearea}
   
\usepackage[utf8]{inputenc} % danske tegn
\usepackage[danish]{babel} % dansk sprogpakke

% Hvis engelsk sprog ønskes, uncomment denne
% \usepackage[english]{babel}

%%% % % % % % % PAKKER % % % % % % % % % % % % % % % % % % % % % % % 
     % % % % %
     %Pakker til fonts, farver, referencer og ligenende er placeret relevant sted i dokumentet.
     % Her findes generelle pakker som bruges overordnet i dokumentet % 
     \usepackage{footnote} % Package used to controle footnote(specially in tables etc.)
     \usepackage{amsmath}
     \usepackage{slantsc}
     
     \usepackage{pifont}
     \usepackage{amssymb}
     \usepackage{textcomp}
    
        %  \usepackage{float} % http://en.wikibooks.org/wiki/LaTeX/Floats,_Figures_and_Captions
                 \usepackage{multirow} %sammenfletning af r�kker
                % \usepackage{subfig}
                 \usepackage{longtable} % tabeller der g�r over flere r
                  \usepackage{geometry}
         \usepackage[section]{placeins} %Gør det muligt at bruge \FloatBarrier
   
    
    \usepackage[compatibility=false]{caption}
    \captionsetup{font=small,labelfont=bf,textfont=it}
    \usepackage{graphicx}
    \usepackage{afterpage}
    
    
    

    \usepackage[draft]{fixme} %hvor 'options' bl.a. inkluderer final, draft, danish, inline og margin
    
   %\usepackage[pdftex,plainpages=false,pdfpagelabels,colorlinks=true,citecolor=black,linkcolor=black,breaklinks]{hyperref}
   
  
 \usepackage{rotating}
   
  
   \usepackage{eso-pic} % til inds�ttelse af forside.
   % \usepackage{clock} % Til at lave et sejt anaogt ur 

   \usepackage{rotating} % rotate stuff(EVERYTHING!11)
   %\usepackage{lipsum}  % Use \lipsum to generate Lorem Ipsum text.
      
   
   
% % % % % % % % % % % % % % % MATH STUFF % % % % % % % % % % % % % %
    %\usepackage{algorithmic}

    %\usepackage{harpoon} %Tilf�jer harpun-pile over st�rrelser i math, m.m.
 
    %\usepackage{pst-plot} % For axes   
    %\usepackage{breqn
    %\usepackage{fourier}  %................... Roman+math - Utopia
    
  
    %\usepackage{nomentbl} % http://ctan.mackichan.com/macros/latex/contrib/nomentbl/nomentbl.pdf

    %\usepackage{psfrag} % insæt tag, mest brugt i forhold til grafer

   
   

%%% DIVERSE CUSTOMS PAKKET NED I FILER FOR OVERBLIK
    %


%Dette hax g�r s�ledes at man kan lave en itemize inden i en tabel.

\usepackage{mdwlist}
\makeatletter
\def\noVSpace{\@minipagetrue}
\newenvironment{tabItemize}{%
  \@minipagetrue%
  \begin{itemize*}%
}{\vspace{-\normalbaselineskip}%
  \end{itemize*}}
\makeatother

\newcolumntype{d}[1]{D{.}{.}{#1}}


%___________SEMI-t�ttere itemize
%eks:
%\begin{itemize} \semitightlist
%   \item
%\end{itemize}


\newcommand{\semitightlist}{%
\setlength{\itemsep}{0.2cm} \setlength{\parskip}{1pt}}


%Inds�ttelse af NAnote
\usepackage{ifthen}
\newcommand{\clearemptydoublepage}{\newpage{\pagestyle{empty}%
  \cleardoublepage{}}}
\newcommand{\NAfinalversion}{false}
\newcommand{\NAcompileimages}{true}
%% IN TEXT COMMENTING COMMANDS
%% To remove comments redefine \NAfinalversion to true in NA_RepPreamble
\ifthenelse{\equal{\NAfinalversion}{false}}
  { \newcommand\Note[1]{\textcolor{red}{\large\texttt{(NOTE: #1)}}} }
  { \newcommand\Note[1]{} }

%%%% Nomenklatur under formler
%\newcommand\nomenklatur[1]{%
%\begin{table}[htb]
%\centering
%\begin{tabular}{p{0.80\textwidth}}
%\begin{center}\small \textit{#1}\end{center}
%\end{tabular}
%\end{table}}


%___Til at vise programkode
\usepackage{listings}
\lstset{extendedchars=true, basicstyle=\ttfamily,keywordstyle=\normalfont\ttfamily, columns=flexible, numbers=left, numberstyle=\tiny, breaklines, breakatwhitespace=true, language=Matlab, morecomment=[l][\color{blue}]{\%}}

\renewcommand\ttdefault{txtt}

\newcommand\uu[1]{\underline{\underline{#1}}}
\newcommand\RA{\qquad \Rightarrow}
\newcommand\qqquad{\qquad \qquad}
\newcommand\qqqquad{\qqquad \qqquad}
\newcommand\qqqqquad{\qqqquad \qqqquad}
\newcommand\degree{^{\circ}}
\newcommand\f{\\[0.3cm]}
\renewcommand{\Re}{\operatorname{Re}}
\renewcommand{\Im}{\operatorname{Im}}
\newcommand\units[1]{\left[{#1}\right]}

%%%___Morten Kjelds hax
%Inds�ttelse af fede gr�ske vektorer
\newcommand{\gve}[1]
{\textrm{\boldmath ${#1}$ \unboldmath}\!}
%Inds�ttelse af vektor
\newcommand{\ve}[1]
{\mathbf{#1}}
%Inds�ttelse af vektor med streg over
\newcommand{\ove}[1]
{\overline\mathbf{#1}}
%Inds�ttelse af �, � og � i formeludtryk
\newcommand{\�}
{\textit{\textrm{�}}}
\newcommand{\�}
{\textit{\textrm{�}}}
\newcommand{\�}
{\textit{\textrm{�}}}

\definecolor{shadecolor}{gray}{.91}


    %
%%%%%%%%%%%%%%%%%%%%%%%%%%%%%%%%%%%%%%%%%%%%%%%%%%%%%%%%%%%%%%%%%%%%%%%%%%%%%
%                              Custom functions                             %
%                         For general use in projects                      %
%%%%%%%%%%%%%%%%%%%%%%%%%%%%%%%%%%%%%%%%%%%%%%%%%%%%%%%%%%%%%%%%%%%%%%%%%%%%%

% For use in definition of new commands:
\usepackage{xspace}           % \newcommand{\cmd}{text\xspace} ensures the correct spacing between text and other words/characters.

% For cellheight adjustments in nomenklatur-function
\usepackage{cellspace}
    \addtolength\cellspacetoplimit{0pt}
    \addtolength\cellspacebottomlimit{0pt}


%
%% Custom "paragraph" function for making line-seperated paragraphs (e.g. double newline)
%

% Make double line break with no line indent (for changing the way paragraphs are made)
\newcommand{\newpar}{$\,$ \\ $\,$ \\}
% Alias of \newpar
\newcommand{\newparagraph}{\newpar{}}

\newcommand{\subsubsubsection}[1]{%
\newpar\paragraph{#1}$\,$\\%
}


%
%% Function for creating horizontal lines between points in a recipe-like list
%
\newcommand{\recipeline}{%
\vspace{2pt}
\hrule
\vspace{2pt}
}
%
%% Function for creating a boxed, centered equation  behaving like the equation environment
%
\newcommand{\boxedeq}[2]{%
	\[\fbox{%
			\addtolength{\linewidth}{-2\fboxsep}%
			\addtolength{\linewidth}{-2\fboxrule}%
			\begin{minipage}{#2}%
				\bigskip%
				\[#1\]%
				\hspace{1cm}%
			\end{minipage}%
	}\]%
			\addtolength{\linewidth}{-2\fboxsep}%
			}
			%
			%% Function for creating a boxed, centered equation  behaving like the equation environment
			%
			\newcommand{\boxedeqlabel}[3]{%
				\[\fbox{%
			\addtolength{\linewidth}{-2\fboxrule}%
			\begin{minipage}{#2}%
				\bigskip%
				\begin{equation}#1\end{equation}%
				\label{#3}%
				\hspace{1cm}%
			\end{minipage}%
	}\]%
}

%
%% Environment for displaying a 0th, 1st or 2nd order tensor in matrix format
%
\newenvironment{tensor}
{
	\begin{pmatrix}
}
{
    \end{pmatrix}
}

%
%% For displaying vectors
%
\newenvironment{vect}
{
	\begin{Bmatrix}
}
{
    \end{Bmatrix}
}

% VECTOR IS DEFINED
\newcommand{\vectq}[1]{\ensuremath{%
%	\{#1\}
\text{\overrightharp{$#1$}}
}}

% UNIT VECTOR IS DEFINED
\newcommand{\unitvecq}[1]{\ensuremath{%
\widehat{#1}
}}

% TIME INVARIANT IS DEFINED
\newcommand{\tInvar}[1]{
\widetilde{#1}
}
%
%% For displaying matrices
%
\newenvironment{matr}
{
	\begin{bmatrix}
}
{
    \end{bmatrix}
}
\newcommand{\matrq}[1]{\ensuremath{%
%	[#1]
\underline{#1}
}}
\renewcommand{\Re}[1]{
%\text{Re}\Big\langle#1\Big\rangle%
\text{Re}\left\{#1\right\}
}
\renewcommand{\Im}[1]{
%\text{Im}\Big\langle#1\Big\rangle%
\text{Im}\left\{#1\right\}
}
%
%% Function for creating small introductions for every new chapter (italic text)
%
\newcommand{\chapintro}[1]{%
    \textit{#1} \\ \linebreak%
    \noindent%
}

%
%% Function for creating small introductions for every new section (italic text)
%
\newcommand{\secintro}[1]{%
    \textit{#1} \\ \linebreak%
    \noindent%
}

%
%% Function for creating isolated remarks (indented italic text)
%
\newcommand{\remark}[1]{%
	\begin{center}%
		\parbox{0.8\textwidth}{\textit{#1}}%
	\end{center}%
    \noindent%
}

%
%% Function for creating isolated remarks (indented italic text)
%
\newcommand{\makequote}[2]{%
	\begin{center}%
		\parbox{0.8\textwidth}{"\textit{#1}" #2}%
	\end{center}
    \noindent%
}

%%% Function for using \verytightlist - not very pretty!
%\newcommand{\verytightlist}{%
%    \setlength{\itemsep}{0.05cm} \setlength{\parskip}{1pt}}
%
%
%%% Function for almost using \tightlist without using the memoir-class
%\newcommand{\allmosttightlist}{%
%    \setlength{\itemsep}{0.1cm} \setlength{\parskip}{1pt}}
%
%
%%% Function for using \semitightlist
%\newcommand{\semitightlist}{%
%    \setlength{\itemsep}{0.2cm} \setlength{\parskip}{1pt}}
%

%
%% Comments (shown in document outer margin)
%
\renewcommand{\comment}[2]{%
    \textcolor{blue}{\textbf{*}} \marginpar{%
        \fcolorbox{white}{blue}{\parbox{3.5cm}{%
            \color{white} \small \sffamily \textbf{#1:} #2}%
        }%
    }%
}
%\newcommand{\comment}[2]{} % Use for hiding comments

%
%% Fixmes (shown in document outer margin)
%
\renewcommand{\fixme}[2]{%
    \textcolor{red}{\textbf{!}} \marginpar{%
        \fcolorbox{white}{red}{\parbox{3.5cm}{%
            \color{white} \small \sffamily \textbf{FIXME:} #2 (#1)}%
        }%
    }%
}
%\newcommand{\fixme}[2]{} % Use for hiding fixmes


%
%% Pretext style headers/footers (for use before \mainmatter}
%% NB: Use only for report/book style in documentclass
%
\newenvironment{pretext}
{
	% Don't use
	\pagestyle{plain}
}
{
	\cleardoublepage
	\pagestyle{fancy}
}


%%% Function for using \bf to write in bold
\renewcommand{\bf}{%
    \bfseries%
}

%%% Function for using \it to write in italic
\renewcommand{\it}{%
    \itshape%
}


%% Home made commands by Rene
\newcommand{\abs}[1]{\lvert#1\rvert}
\newcommand{\sgn}[1]{\text{sgn}\left(#1\right)} %Sign function
\newcommand{\nrcs}[1]{\: \put(5.5,3.5){\circle{14}}#1 \,} %Laver en cirkel omkring et to cifferet tal, nummer cirkel stor fx \nrcs{15}
\newcommand{\nrc}[1]{\: \put(3,3.5){\circle{12}}#1 \,} %Laver en cirkel omkring et �t cifferet tal, nummer cirkel fx \nrc{5}
\newcolumntype{R}[1]{>{\raggedleft\arraybackslash}p{#1}} %R{2cm} laver en right orienteret s�jle med en bredde p� 2cm






%%%%%%%%%%%%%%%%%%%%%%%%%%%%%%%%%%%%%%%%%%%%%%%%%%%%%%%%%%%%%%%%%%%%%%%%%%%%%
%                              Custom functions                             %
%                       Primarily for this project; I6                      %
%%%%%%%%%%%%%%%%%%%%%%%%%%%%%%%%%%%%%%%%%%%%%%%%%%%%%%%%%%%%%%%%%%%%%%%%%%%%%

%
%% Adds vertical space around \hline in tabular, T = top
%
\newcommand\T{%
    \rule{5pt}{5ex}%
}

%
%% Adds vertical space around \hline in tabular, B = buttom
%
\newcommand\B{%
    \rule{0pt}{2.6ex}%
}

%
%% Nomenclature below equations
%
\newcommand\nomenklaturCellHeight{%
\includegraphics[scale=1]{styles/nomenklaturCellHeight.PNG}%
}
\renewcommand\nomenklaturCellHeight{}


\newcommand\nomenklatur[2]{%
\nomenclature{$#1$}{#2.}{}{}%
\begin{minipage}[top]{0.7\textwidth}\flushleft%\renewcommand\arraystretch{1.25}% (Value=1.0 is for standard spacing}%
\begin{tabular}{S{p{1.0cm}}S{p{1.3cm}} Sl}%
{$\,$} & \ensuremath{#1} &\nomenklaturCellHeight\parbox{12cm}{\footnotesize{#2.}}%
\end{tabular}%
\end{minipage}%
%}
\\ \noindent%
}

%
%% Start nomenclature below equations - use this command for the first line of each nomenclature-block.
%
\newcommand\nomenklaturstart[2]{%
\nomenclature{$#1$}{#2.}{}{}%
\begin{minipage}[top]{0.7\textwidth}\flushleft%\renewcommand\arraystretch{1.25}% (Value=1.0 is for standard spacing}%
\begin{tabular}{>{\setlength{\parindent}{-0.2cm}}S{p{1.0cm}}S{p{1.3cm}} Sl}
where:\nomenklaturCellHeight & \ensuremath{#1} &\nomenklaturCellHeight\parbox{12cm}{\footnotesize{#2.}}%
\end{tabular}%
\end{minipage}%
%}
\\ \noindent%
}

%%%%%%%%%%%%%%%%%%%%%%%%%%%%%%%%%%%%%%%%%%%%%%%%%%%%%%%%%%%
%%figur nomenklatur
%%%%%%%%%%%%
%\newcommand\ehnomenklatur[3]{%
%\noindent
%\nomenclature{$#1$}{#2}{}{}%
%\begin{minipage}[top]{0.7\textwidth}\flushleft%\renewcommand\arraystretch{1.25}% (Value=1.0 is for standard spacing}%
%\begin{tabular}{S{p{1.0cm}} S{p{1.3cm}} Sl}%
%{$\,$} & \ensuremath{#1} &\nomenklaturCellHeight\parbox{8cm}{\footnotesize{#2}} \ensuremath{\footnotesize{#3}}%
%\end{tabular}%
%\end{minipage}%
%%}
%\\ \noindent%
%}
%
%%
%%% Start nomenclature below equations - use this command for the first line of each nomenclature-block.
%%
%\newcommand\ehnomenklaturstart[3]{%
%\noindent
%\nomenclature{$#1$}{#2}{}{}%
%\begin{minipage}[top]{0.7\textwidth}\flushleft%\renewcommand\arraystretch{1.25}% (Value=1.0 is for standard spacing}%
%\begin{tabular}{>{\setlength{\parindent}{-0.2cm}}S{p{1.0cm}}S{p{1.3cm}} Sl}
%$\quad$ \\
%where:\nomenklaturCellHeight & \ensuremath{#1} &\nomenklaturCellHeight\parbox{8cm}{\footnotesize{#2}} \ensuremath{\footnotesize{#3}}%
%\end{tabular}%
%\end{minipage}%
%%}
%\\ \noindent%
%}


%
%% Command for definition of new terms in text
%
\newcommand\term[1]{%
    \emph{#1}%
}

%% Command for creating a double-bar over a matrix symbol
\newcommand\dbar[1]{%
    \ensuremath{\bar{\bar{#1}}}%
}

%% Command for definition of planar cross product
\newcommand\planarCross{%
%    \ensuremath{\underline{\times}}%
%    \ensuremath{\dtimes}%
\begin{turn}{90}\ensuremath{\ltimes}\end{turn}
}



    \usepackage{todonotes}


% new commands for individually colored todo 
\newcommand{\todola}[2]{
\todo[#1, color=blue!40]{#2 \\- Lasse}
}
\newcommand{\todokr}[2]{
\todo[#1, color=red!40]{#2 \\- Kristoffer}
}
    %%%%%%%%%%%%%%%%%%%%%%%%%%%%%%%%%%%%%%%%%%%%%%%%%%%%%%%%%%%%%%%%%%%%%%%%%%%%%
%                            Custom for funktioner                          %
%                     			      	        		                    %
%%%%%%%%%%%%%%%%%%%%%%%%%%%%%%%%%%%%%%%%%%%%%%%%%%%%%%%%%%%%%%%%%%%%%%%%%%%%%


% Package for wrapping figures in text
\usepackage{wrapfig}

% Set the path where LaTeX looks for pictures.
\graphicspath{{../figurer/}}

% You need a newsubfloat element to use subcaption
\newsubfloat{figure}

% Command to set caption styles
\captionnamefont{\bfseries\small}
\captiontitlefont{\itshape\small}
\subcaptionlabelfont{\bfseries\small}
\subcaptionfont{\itshape\small}

% When we don't use chapter numbering, 
% the figure numbering are shown like 
% fx 0.3 if we don't use this line.
\counterwithout{figure}{chapter}


% Command \figur{filename}{caption}{label}{width} 
% for inserting a new figure.
\newcommand{\figur}[4]{ 
 \begin{figure}[ht] 
 \centering 
 \includegraphics[width=#4\textwidth]{#1} 
 \caption{#2} 
 \label{#3} 
 \end{figure}
 } 
 
% Command \wrapfigur{filename}{caption}{label}{width}{r/l}
% for inserting a new figure wraped in the text. 
\newcommand{\wrapfigur}[5]{ 
 \begin{wrapfigure}{#5}{#4 \textwidth}
   \begin{center}
    \includegraphics[width=#4\textwidth]{#1}
   \end{center}
   \caption{#2}
   \label{#3} 
 \end{wrapfigure}
}
 
% Command \subfigur{filename1}{filename2}{caption}
% {subcaption1}{subcaption2}{label1}{label2}
% {width1}{width1} for inserting a new figure.
\newcommand{\subfigur}[9]{
\begin{figure}[ht]
\centering
\subbottom[{#4}\label{#6}]%
    {\includegraphics[width=#8\textwidth]{#1}}\hfill
\subbottom[{#5}\label{#7}]%
    {\includegraphics[width=#9\textwidth]{#2}}
\caption{#3}
\label{fig:subfigures}
\end{figure}
}
% Command \truefigure{filename}{caption}{label}
% for inserting a picture without scaling.
\newcommand{\truefigure}[3]{ 
 \begin{figure}[ht] 
 \centering 
 \includegraphics{#1} 
 \caption{#2} 
 \label{#3} 
 \end{figure}
 } 

% Keeps floats in the section. 
% Help control place to put pix
\usepackage[section]{placeins}

% Commands for rightaligning wraped text in a tabular
\newcommand{\rr}{\raggedright}
\newcommand{\tn}{\tabularnewline}

\usepackage{multirow}

%Tabular pakke som Wex har fundet, men som vi ikke bruger alligvel
%\usepackage{tabu}

% Include code pieces from files
% Documentation:
% ftp://ftp.tex.ac.uk/tex-archive/macros/latex/contrib/listings/listings.pdf
\usepackage{listings}


%Package with enumirate command with extra options (used within cells of tabular)
\usepackage[inline]{enumitem}
    \usepackage{geometry}
\usepackage{amsmath}
\usepackage[some]{background}


\definecolor{titlepagecolor}{cmyk}{1,.60,0,.40}

\DeclareFixedFont{\bigsf}{T1}{phv}{b}{n}{1.2cm}

\backgroundsetup{
scale=1,
angle=0,
opacity=1,
contents={\begin{tikzpicture}[remember picture,overlay]
 \path [fill=ThemeColor] (-0.5\paperwidth,5) rectangle (0.5\paperwidth,10);  
\end{tikzpicture}}
}
\makeatletter                       
\def\printauthor{%                  
    {\large \@author}}              
\makeatother
\author{%
20104172, Lasse Bue Svendsen \\
\vspace*{0.35cm}
\underline{\hspace{5.5cm}}
\\
\vspace*{0.75cm}
201270860, Kristoffer Sloth Gade \\
\vspace*{0.35cm}
\underline{\hspace{5.5cm}}
\\



\vspace*{2cm}
Vejleder: Peter Ahrendt
\vspace*{0.35cm}
\underline{\hspace{5.5cm}}
}



    
    
    
% % % % % % % % % % % % % % ********% % %Generelt design % % % % % % % % % % % % % % % % % % %
 
%\usepackage[table,dvipsnames]{xcolor} % 'table' er i stedet for \usepackage{colortbl} og 'dvipsnames' er et farveskema hvor man angiver farver med navne
 
\usepackage{xcolor} % farver kan indsætttes i henhold til RGB med definecolor

%Valg af temafarve % Denne farve bruges i chapterstyle %
\definecolor{ThemeColor}{RGB}{224, 32, 28}


% Valg af font %
\usepackage[sc]{mathpazo}
\linespread{1.05}         % Palatino needs more leading (space between lines)
\usepackage[T1]{fontenc}


% Pakker inkludere overordnede designvalg. Pakkerne er valgt for overskuelighed %
	 %%%%%%%%%%%%%%%%%%%%%%%%%%%%%%%%%%%%%%%%%%%%%%%%%%%%%%%%%%%%%%%%%%%%%%%%%%%%%
%                             Chapter and Section                           %
%                     			      Style	       		                    %
%%%%%%%%%%%%%%%%%%%%%%%%%%%%%%%%%%%%%%%%%%%%%%%%%%%%%%%%%%%%%%%%%%%%%%%%%%%%%


% % % % Redefined so that header will show even on chapter start. % % % %
\makeatletter
    \let\stdchapter\chapter
    \renewcommand*\chapter{%
    \@ifstar{\starchapter}{\@dblarg\nostarchapter}}
    \newcommand*\starchapter[1]{%
        \stdchapter*{#1}
        \thispagestyle{fancy}
        \markboth{\MakeUppercase{#1}}{}
    }
    \def\nostarchapter[#1]#2{%
        \stdchapter[{#1}]{#2}
        \thispagestyle{fancy}
    }
\makeatother



\usepackage{titlesec} % Pakke til opsætning af chapter og sektioner


% % defining new font for the chapternumber. Calling it FuckingHuge 
\newcommand{\FuckingHuge}{% Needs a 'stretchable' font
      \usefont{\encodingdefault}{\rmdefault}{b}{n}%
      \fontsize{31}{45}\normalfont\color{ThemeColor}} %60 80


%%% TAL I OVERSKRIFTER, både chapters og sections. 
    \maxsecnumdepth{subsubsection} %subsubsection er dybdeniveau for numering.
   	\setsecnumdepth{subsubsection}


  
 % % % % CHAPTER OPSÆTNING % % % % % % % % % % % % % % % % % % % % %

%Hang thoose numbers! I margin med tallene for både sections og chapters! En lækker chapter style!
\chapterstyle{hangnum}
 
%Modificering af ovenstående style
%Placering af chapter overskriften på siden i forhold til sidens start %
\renewcommand*{\chapterheadstart}{\vspace*{-20pt}}


% FARVE PÅ CHAPTER % %
%TItel farve og font valg
\renewcommand*{\chaptitlefont}{\normalfont\FuckingHuge\color{ThemeColor}}
%Numering farve og font valg
\renewcommand{\chapnumfont}{\normalfont\FuckingHuge\color{ThemeColor}}



% % % % SECTION OPSÆTNING % % % % % % % % % % % % % % % % % % % %



\hangsecnum % vi smidder numering ud i marken




%Vi benytter tema farven til numeringen
\renewcommand\thesection{\color{ThemeColor}\thechapter.\arabic{section}}




    
    


	 \input{Styles/header.tex}
	 

% Setup the margins:
\setlrmarginsandblock{1in}{*}{1.5}
\setulmarginsandblock{3cm}{4cm}{*}



% Setup header and footer: %
% \setheadfoot{headheight}{footskip}
% \setheaderspaces{headdrop}{headsep}{ratio}
\setheadfoot{45pt}{1cm}
\setheaderspaces{1cm}{*}{*}
\checkandfixthelayout
% Se dette pdf for dokmentaton:
% www.tug.dk/_media/foredrag/2003-12-03/slides-src.pdf




 \checkandfixthelayout[nearest]
    \checkthelayout
    \fixthelayout
    \baselineskip \topskip



% % % % % % % % % % % % % % % % % % % %***** % % % % % % % % % % % % % % % % % % % % % % % % %

%BibTex ops�tning
\usepackage[semicolon,authoryear,square]{natbib}

\bibliographystyle{plainnat}
%%%%%%%%%%%%%%%%%%%%%%%%%%%%%%%%%%%%%%%%%%%%%%%%%%%%%%%%%%%%%%%%%%%%%%%%%%%%%%%%%%%%%%%%%%%%%%

   
    
% % Smart listing function. Documentation: http://texblog.org/tag/makenomenclature/
\usepackage{nomencl}
\makenomenclature 



% % % % % % % % % % % % % % % % Reference opsætning % % % % % % % % % % % % % % % % % % % % % % %

\usepackage{varioref} % for references, http://www.ctex.org/documents/packages/bibref/varioref.pdf
   
   
    
% External reference package, zref makes the references into hyperlinks.
\usepackage{url}
\usepackage{hyperref}
%\usepackage{zref-xr}
\usepackage{xr}
%Reference til ekstern Rapport.
%\externaldocument[Rapport-]{../Rapport/Rapport}

% % % % % % % % % % % % % % % % % % % % % % % % % % % % % % % % % % % % % % % % % % % % % % % % % % %

%\usepackage{showlabels}


% footruleskip is allready defined in memoir, so we have to undefine it, in order to include fancyhdr for our footer and header.
\let\footruleskip\undefined
\usepackage{fancyhdr}% For making fancy headers http://ctan.org/pkg/fancyhdr




\title{
\Huge\huge\bfseries\color{ThemeColor}Projektdokumentation - 3. semesterprojekt\\
\Huge\huge\bfseries\color{ThemeColor}Automatisk miljøstyring af et orangeri\\
\vspace{1em}
\large\color{ThemeColor}Ingeniørhøjskolen Aarhus Universitet}
\large\date{20. december 2013}


\AtBeginDocument{\renewcommand\contentsname{Indholdsfortegnelse}}

% % % % % % % % % % % % % Inkluderinger af ønskede kapitler % % % % % % % % % % % % %

\begin{document}
\thispagestyle{empty} % No header and footer on this page

\BgThispage
\newgeometry{left=1cm,right=4cm}
\vspace*{2cm}
\noindent
\textcolor{white}{\bigsf Mingle \\ Accepttestspecifikation}
\vspace*{2.5cm}\par
\noindent
\begin{minipage}{0.45\linewidth}
    \begin{flushright}
        \printauthor
    \end{flushright}
\end{minipage} \hspace{30pt}
%
\begin{minipage}{0.02\linewidth}
    \rule{1pt}{275pt}
\end{minipage} \hspace{-10pt}
%
\begin{minipage}{0.6\linewidth}
\vspace{5pt}
 

\includegraphics[width=11cm]{Logo}
  
Gruppe 2 \newline

Dato 27/5 - 2014 \newline

Ingeniørhøjskolen Aarhus Universitet \newline





\end{minipage}
\restoregeometry


\mbox{} % Insert a blank page so, toc appears on an odd pagenumber
\thispagestyle{empty} % No header and footer on this page
\newpage

\thispagestyle{empty} % No header and footer on this page
\tableofcontents*
\thispagestyle{empty} % No header and footer on this page

% Setup header and footer
\pagestyle{fancy}
\fancyhf{} % Clear all header an footer fields 
\renewcommand{\headrulewidth}{0pt} % Delete headruler
\fancyhead[R]{\includegraphics[height=40pt]{header}} % Insert picture in left side of header
%\fancyhead[R]{\rightmark} % insert chapter title in right side of header
\fancyfoot[LE,RO]{\thepage} % insert pagenumber on right side of even page, left side of odd page 
%\setcounter{page}{1}

\chapter{Introduktion}
 \label{chp:introduktion}
\section{Indledning}
Dette dokument indeholder projektets acceptest. Inden overlevering af projektets produkt til kunden udføres denne accepttest sammen med produktejeren eller en repræsentant for denne. Accepttesten udføres efter at samtlige af systemets enheder har gennemført relevante enhedstest og integrationstest. 

\subsection{Formål}
Formålet med dette dokument er:
\begin{itemize}
\item At teste hvorvidt systemet opfylder funktionelle og ikke-funktionelle krav.
\end{itemize}

\subsection{Refereret dokumentation}
\begin{itemize}
\item Kravspecifikation
\end{itemize}

\subsection{Versionshistorik}
\begin{tabular}{p{2cm}p{2cm}p{2cm}p{8cm}}
\textbf{Vers.} & \textbf{Init.} & \textbf{Dato} & \textbf{Beskrivelse} \\
\hline
1.0 & SØK & 15/2 2014 & Tomt dokument inkl. standard header oprettet \\
2.0 & SØK & 19/2 2014 & Accepttest tilføjet for funktionelle krav \\
2.1 & SØK & 24/2 2014 & Accepttest rettet i henhold til ændringer i kravspecifikationer \\
2.2 & SØK & 21/5 2014 & Klargøring og opsætning af den endelige accepttest \\
2.3 & SØK & 25/5 2014 & Færdiggjort til den endelige acceptest \\
\hline

\hline
\end{tabular}

\vspace{\fill}
\begin{tabular}{ p{12cm}  p{3cm} }
\hline
Godkendt af: & Dato
\end{tabular}
\chapter{Testspecifikation}
 \label{chp:testspecifikation}
 

\section{Testopstilling}

\subsection{Enheder som anvendes i testen:}

\begin{itemize}
\item Brugers enhed defineres som en Nokia Lumia 925 kørende Windows Phone 8.
\item Festivalsgæsts enhed defineres som en Nokia Lumia 520 kørende Windows Phone 8.
\end{itemize}


Nedenfor ses den fysiske testopstilling bestående af en Nokia Lumia 925 og en Nokia Lumia 520
\figur {spil2.jpg} {Testopstilling}{Fig:testopstilling}{0.68} 


\subsection{Opsatte testbrugere:}
Følgende bruger er oprettet på Facebook ved testent begyndelse. Igennem forløbet ville disse Facebook brugere benyttes.
\begin{itemize}
\item Bruger benytter Facebookprofilnavnet: Jens Testrup
\begin{itemize}
\item Brugernavn: hauswow@gmail.com
\item Adgangskode: semesterprojekt
\end{itemize}
\item Festivalgæst benytter Facebookprofilnavnet Ulla Testrup
\begin{itemize}
\item Brugernavn: mingleprojekt@gmail.com
\item Adgangskode: semesterprojekt
\end{itemize}
\end{itemize}

\subsection{Nulstilling af point}
Begge testbrugers optjente point nulstilles ved testens start i databasen.


\section{Godkendelseskriterier}
Acceptesten er afsluttet, når samtlige test cases er gennemført og godkendt.

Hvis fejl opstår under acceptest som umuliggør fortsættelsen af acceptesten afbrydes denne.

Hvis der opstår fejl i de enkelte test cases, men acceptesten fortsat kan fortsætte, så underkendes den enkelte test case. Accepttesten fortsættes herefter.





\chapter{Test af funktionelle krav}
 \label{chp:testfunktionelle}

\subsection{Introduktion}
Der arbejdes iterativt i udviklingen af projektet, og samtlige test cases for de udspecificerede kravspecifikationer opstilles således løbende for systemet.  


\section{Test: UC1 Login}

\subsection{UC1 Hovedforløb }


\begin{longtable}{| p{0.7cm}  | p{3cm}  | p{4cm} |  p{3cm}  | p{3cm}  |}
\hline
Step & Handling & Forventet Resultat & Test & Resultat \\
\hline
1 & Bruger åbner mobilapplikationen  & Mobilapplikationen informerer om krav til forbindelse med brugers facebook & Aflæses på brugers enhed & \FuckingHuge{\checkmark} \\
\hline
2 & Bruger vælger at logge ind med Facebook  & Mobilapplikationen fremviser mulighed for at logge ind med Facebook & Aflæses på brugers enhed & \FuckingHuge{\checkmark} \\
\hline
3 & Bruger indtaster brugernavn: 'hauswow@ gmail.com' og adgangskode: 'semesterprojekt' og bekræfter & hovedmenuen fremvises på mobilapplikationen  & Aflæses på brugers enhed  & \FuckingHuge{\checkmark} \\
\hline
\end{longtable}


\subsection{UC1: [Extension 1: Bruger har tidligere logget ind]}

\begin{longtable}{| p{0.7cm}  | p{3cm}  | p{4cm} |  p{3cm}  | p{3cm}  |}
\hline
Step & Handling & Forventet Resultat & Test & Resultat \\
\hline
1 & Bruger åbner mobilapplikationen  & Applikation informerer om krav til forbindelse med brugers facebook & Krav aflæses på brugers enhed & \FuckingHuge{\checkmark} \\
\hline
2 & Bruger vælger at logge ind med Facebook  & Hovedmenuen fremvises på mobilappplikation & Aflæses på brugers enhed & \FuckingHuge{\checkmark}\\
\hline
\end{longtable}




\newpage
\section{Test: UC2 Enheder forbindes}



\subsection{UC2 Hovedforløb}

\begin{longtable}{| p{0.7cm}  | p{3cm}  | p{4cm} |  p{3cm}  | p{3cm}  |}
\hline
Step & Handling & Forventet Resultat & Test & Resultat \\
\hline
1 &  Bruger og Festivalgæst vælger Mingle i hovedmenuen &  Mulighed for at vælge at starte et spil og deltage i et spil fremvises  & Aflæses på Brugers og Festivalgæsts enhed & \FuckingHuge{\checkmark} \\
\hline
2 &  Bruger vælger at oprette spil &  Brugers id vises på enheden  & Aflæses på Brugers enhed & \FuckingHuge{\checkmark} \\
\hline
3 &  Festivalgæst vælger at deltage i et spil &  Festivalgæst har mulighed for at indtaste id  & Aflæses på Festivalgæsts enhed & \FuckingHuge{\checkmark} \\
\hline
4 &  Bruger udveksler unikt id med Festivalgæst, og Festivalgæst indtaster Brugers unikke id &  Brugers id er indtastet på Festivalgæsts enhed  & Aflæses på Festivalgæsts enhed & \FuckingHuge{\checkmark} \\
\hline
5 &  Bruger udveksler unikt id med Festivalgæst, og Festivalgæst indtaster Brugers unikke id &  Brugers id er indtastet på Festivalgæsts enhed  & Aflæses på Festivalgæsts enhed & \FuckingHuge{\checkmark} \\
\hline
6 & Festivalgæst accepterer &  Bruger og Festivalgæst informeres om at de er forbundet & Aflæses på Brugers og Festivalsgæsts enhed & \FuckingHuge{\checkmark}\\ 
\hline
7 & Bruger og Festivalgæst accepterer &   Brugers og Festivalgæsts mobilapplikationer fremviser identiske spørgsmål og svarmuligheder. & Identiske spørgsmål og svarmuligheder aflæses på Brugers og Festivalgæsts enheder & \FuckingHuge{\checkmark} \\
\hline
\end{longtable}

\newpage

\subsection{UC2: [Extension 1: Bruger og Festivalgæst har tidligere i samarbejde vundet spørgsmålsspillet]}

\begin{longtable}{| p{0.7cm}  | p{3cm}  | p{4cm} |  p{3cm}  | p{3cm}  |}
\hline
Step & Handling & Forventet Resultat & Test & Resultat \\
\hline
1 &  Bruger og Festivalgæst vælger Mingle i hovedmenuen &  Mulighed for at vælge at starte et spil og deltage i et spil fremvises  & Aflæses på Brugers og Festivalgæsts enhed & \FuckingHuge{\checkmark} \\
\hline
2 &  Bruger vælger at oprette spil &  Brugers id vises på enheden  & Aflæses på Brugers enhed & \FuckingHuge{\checkmark} \\
\hline
3 &  Festivalgæst vælger at deltage i et spil &  Festivalgæst har mulighed for at indtaste id  & Aflæses på Festivalgæsts enhed & \FuckingHuge{\checkmark} \\
\hline
4 &  Bruger udveksler unikt id med Festivalgæst, og Festivalgæst indtaster Brugers unikke id &  Brugers id er indtastet på Festivalgæsts enhed  & Aflæses på Festivalgæsts enhed & \FuckingHuge{\checkmark} \\
\hline
5 & Festivalgæst accepterer &  Bruger og Festivalgæst informeres om at de er forbundet, men at der ikke optjenes point & Aflæses på Brugers og Festivalsgæsts enhed & \FuckingHuge{\checkmark}\\ 
\hline
6 & Bruger og Festivalgæst accepterer &   Brugers og Festivalgæsts mobilapplikationer fremviser identiske spørgsmål og svarmuligheder. & Identiske spørgsmål og svarmuligheder aflæses på Brugers og Festivalgæsts enheder & \FuckingHuge{\checkmark} \\
\hline
\end{longtable}


\subsection{UC2: [Extension 2: Bruger indtaster ikke korrekt id]}


\begin{longtable}{| p{0.7cm}  | p{3cm}  | p{4cm} |  p{3cm}  | p{3cm}  |}
\hline
Step & Handling & Forventet Resultat & Test & Resultat \\
\hline
1 &  Bruger og Festivalgæst vælger Mingle i hovedmenuen &  Mulighed for at vælge at starte et spil og deltage i et spil fremvises  & Aflæses på Brugers og Festivalgæsts enhed & \FuckingHuge{\checkmark} \\
\hline
2 &  Bruger vælger at oprette spil &  Brugers id vises på enheden  & Aflæses på Brugers enhed & \FuckingHuge{\checkmark} \\
\hline
3 &  Festivalgæst vælger at deltage i et spil &  Festivalgæst har mulighed for at indtaste id  & Aflæses på Festivalgæsts enhed & \FuckingHuge{\checkmark} \\
\hline
4 &  Bruger udveksler unikt id med Festivalgæst, og Festivalgæst indtaster '12345' &  '12345' er indtastet på Festivalgæsts enhed  & Aflæses på Festivalgæsts enhed & \FuckingHuge{\checkmark} \\
\hline
5 & Festivalgæst accepterer & Festivalgæst informeres om at der ikke finde en aktiv spiller med dette id & Aflæses på Brugers og Festivalsgæsts enhed & \FuckingHuge{\checkmark} \\ 
\hline
\end{longtable}


\newpage

\section{Test: UC3 Spil nyt spil}

\subsection{UC3 Hovedforløb}

\begin{longtable}{| p{0.7cm}  | p{3cm}  | p{4cm} |  p{3cm}  | p{3cm}  |}
\hline
Step & Handling & Forventet Resultat & Test & Resultat \\
\hline
1 &  Bruger har afsluttet UC2  &  Brugers og Festivalgæsts mobilapplikationer fremviser spørgsmål og svarmuligheder. & Identiske spørgsmål og svarmuligheder aflæses på Brugers og Festivalgæsts enheder & \FuckingHuge{\checkmark} \\
\hline
2 &  Bruger og festivalgæst vælger det øverste svar &  Mobilapplikationerne informerer Bruger og Festivalgæst om korrekt svar. & Afæses på Brugers og Festivalgæsts enheder & \FuckingHuge{\checkmark}\\ 
\hline
3 &  Bruger og Festivalgæst accepterer korrekt svar  &  Brugers og Festivalgæsts mobilapplikationer fremviser spørgsmål og svarmuligheder. & Identiske spørgsmål og svarmuligheder aflæses på Brugers og Festivalgæsts enheder & \FuckingHuge{\checkmark} \\
\hline
4 &  Bruger og festivalgæst vælger det miderste svar &  Applikationerne informerer Bruger og Festivalgæst om korrekt svar. & Afæses på Brugers og Festivalgæsts enheder & \FuckingHuge{\checkmark}\\ 
\hline
5 &  Bruger og Festivalgæst accepterer korrekt svar  &  Brugers og Festivalgæsts mobilapplikationer fremviser spørgsmål og svarmuligheder. & Identiske spørgsmål og svarmuligheder aflæses på Brugers og Festivalgæsts enheder & \FuckingHuge{\checkmark} \\
\hline
6 &  Bruger og festivalgæst vælger det nederste svar &  Applikationerne informerer Bruger og Festivalgæst om vundet spil og at der er optjent point. & Afæses på Brugers og Festivalgæsts enheder & \FuckingHuge{\checkmark}\\ 
\hline
7 & Bruger og Festivalgæst accepterer  & Bruger og Festivalgæst har nu 10 point og hovedmenuen fremvises & Aflæses på Brugers og Festivalgæsts enheder & \FuckingHuge{\checkmark}\\
\hline
\end{longtable}




\subsection{UC3: [Extension 1: Bruger og festivalgæst vælger ikke samme svar]}

\begin{longtable}{| p{0.7cm}  | p{3cm}  | p{4cm} |  p{3cm}  | p{3cm}  |}
\hline
Step & Handling & Forventet Resultat & Test & Resultat \\
\hline
1 &  Bruger har afsluttet UC2  &  Brugers og Festivalgæsts mobilapplikationer fremviser spørgsmål og svarmuligheder. & Identiske spørgsmål og svarmuligheder aflæses på Brugers og Festivalgæsts enheder & \FuckingHuge{\checkmark}\\
\hline
2 &  Bruger vælger det øverste svar og Festivalgæst vælger det nederste svar  & Mobilapplikationen informerer om at bruger og festivalgæst har tabt. & Aflæses på Brugers og Festivalgæsts enheder &  \FuckingHuge{\checkmark} \\
\hline
3 &  Bruger og Festivalgæst accepterer  & Mobilapplikationen fremviser hovedmenuen. & Aflæses på Brugers og Festivalgæsts enheder & \FuckingHuge{\checkmark} \\
\hline
\end{longtable}

\subsection{UC3: [Extension 2: Bruger annullerer spørgsmålsspillet]}

\begin{longtable}{| p{0.7cm}  | p{3cm}  | p{4cm} |  p{3cm}  | p{3cm}  |}
\hline
Step & Handling & Forventet Resultat & Test & Resultat \\
\hline
1 &  Bruger har afsluttet UC2  &  Brugers og Festivalgæsts mobilapplikationer fremviser spørgsmål og svarmuligheder. & Identiske spørgsmål og svarmuligheder aflæses på Brugers og Festivalgæsts enheder & \FuckingHuge{\checkmark} \\
\hline
2 &  Bruger annullerer spørgsmålsspillet  & Applikationerne informerer Bruger og Festivalgæst om, at spillet er annulleret. & Aflæses på Brugers og Festivalgæsts enheder &  \FuckingHuge{\textdiv}  \\
\hline
\end{longtable}

\newpage

\subsection{UC3: [Extension 3: Festivalgæst annullerer spørgsmålsspillet]}

\begin{longtable}{| p{0.7cm}  | p{3cm}  | p{4cm} |  p{3cm}  | p{3cm}  |}
\hline
Step & Handling & Forventet Resultat & Test & Resultat \\
\hline
1 &  Bruger har afsluttet UC2  &  Brugers og Festivalgæsts mobilapplikationer fremviser spørgsmål og svarmuligheder. & Identiske spørgsmål og svarmuligheder aflæses på Brugers og Festivalgæsts enheder &  \FuckingHuge{\checkmark} \\
\hline
2 &  Festivalgæst annullerer spørgsmålsspillet  & Applikationerne informerer Bruger og Festivalgæst om, at spillet er annulleret. & Aflæses på Brugers og Festivalgæsts enheder &   \FuckingHuge{\textdiv} \\
\hline
\end{longtable}



\subsection{UC3: [Extension 4:Bruger og Festivalgæst har tidligere i samarbejde vundet spørgsmålsspillet]}

\begin{longtable}{| p{0.7cm}  | p{3cm}  | p{4cm} |  p{3cm}  | p{3cm}  |}
\hline
Step & Handling & Forventet Resultat & Test & Resultat \\
\hline
1 &  Bruger har afsluttet UC2  &  Brugers og Festivalgæsts mobilapplikationer fremviser spørgsmål og svarmuligheder. & Identiske spørgsmål og svarmuligheder aflæses på Brugers og Festivalgæsts enheder & \FuckingHuge{\checkmark} \\
\hline
2 &  Bruger og festivalgæst vælger det øverste svar &  Mobilapplikationerne informerer Bruger og Festivalgæst om korrekt svar. & Afæses på Brugers og Festivalgæsts enheder & \FuckingHuge{\checkmark} \\ 
\hline
3 &  Bruger og Festivalgæst accepterer korrekt svar  &  Brugers og Festivalgæsts mobilapplikationer fremviser spørgsmål og svarmuligheder. & Identiske spørgsmål og svarmuligheder aflæses på Brugers og Festivalgæsts enheder & \FuckingHuge{\checkmark} \\
\hline
4 &  Bruger og festivalgæst vælger det miderste svar &  Applikationerne informerer Bruger og Festivalgæst om korrekt svar. & Afæses på Brugers og Festivalgæsts enheder & \FuckingHuge{\checkmark} \\ 
\hline
5 &  Bruger og Festivalgæst accepterer korrekt svar  &  Brugers og Festivalgæsts mobilapplikationer fremviser spørgsmål og svarmuligheder. & Identiske spørgsmål og svarmuligheder aflæses på Brugers og Festivalgæsts enheder & \FuckingHuge{\checkmark}  \\
\hline
6 &  Bruger og festivalgæst vælger det nederste svar &  Applikationerne informerer Bruger og Festivalgæst om vundet spil og at der er optjent point. & Afæses på Brugers og Festivalgæsts enheder & \FuckingHuge{\checkmark} \\ 
\hline
7 & Bruger og Festivalgæst accepterer  & Bruger og Festivalgæst har nu 10 point og hovedmenuen fremvises & Aflæses på Brugers og Festivalgæsts enheder & \FuckingHuge{\checkmark} \\
\hline
8 & Bruger og Festivalgæst accepterer  & Bruger og Festivalgæst har nu 0 point og hovedmenuen fremvises & Aflæses på Brugers og Festivalgæsts enheder & \FuckingHuge{\checkmark} \\
\hline
\end{longtable}


\newpage




\section{Test: UC4 Aflæs point}

\subsection{UC4 Hovedforløb}

\begin{longtable}{| p{0.7cm}  | p{3cm}  | p{4cm} |  p{3cm}  | p{3cm}  |}
\hline
Step & Handling & Forventet Resultat & Test & Resultat \\
\hline
1 & Bruger aflæser antal optjente point & Mobilapplikationen fremviser 10 point i hovedmenuen & 10 point aflæses på Brugers enheder &  \FuckingHuge{\checkmark} \\
\hline
\end{longtable}



\section{Test: UC5 Indløs point}

\subsection{UC5 Hovedforløb}

\begin{longtable}{| p{0.7cm}  | p{3cm}  | p{4cm} |  p{3cm}  | p{3cm}  |}
\hline
Step & Handling & Forventet Resultat & Test & Resultat \\
\hline
1 &   &  & & \FuckingHuge{\textdiv}  \\
\hline
\end{longtable}

\subsection{UC5: [Extension 1: Bruger har ikke nødvendigt antal point]}

\begin{longtable}{| p{0.7cm}  | p{3cm}  | p{4cm} |  p{3cm}  | p{3cm}  |}
\hline
Step & Handling & Forventet Resultat & Test & Resultat \\
\hline
1 &  &  & & \FuckingHuge{\textdiv} \\
\hline
\end{longtable}



\section{Test: UC6 Aflæs historik}

\subsection{UC6 Hovedforløb}

\begin{longtable}{| p{0.7cm}  | p{3cm}  | p{4cm} |  p{3cm}  | p{3cm}  |}
\hline
Step & Handling & Forventet Resultat & Test & Resultat \\
\hline
1 & &  & & \FuckingHuge{\textdiv}  \\
\hline
\end{longtable}

\subsection{UC6: [Bruger har ikke haft nogen sammenkoblinger]}

\begin{longtable}{| p{0.7cm}  | p{3cm}  | p{4cm} |  p{3cm}  | p{3cm}  |}
\hline
Step & Handling & Forventet Resultat & Test & Resultat \\
\hline
1 &  & &  & \FuckingHuge{\textdiv} \\
\hline
\end{longtable}


\newpage

\section{Test: UC7 Log ud}

\subsection{UC7 Hovedforløb}

\begin{longtable}{| p{0.7cm}  | p{3cm}  | p{4cm} |  p{3cm}  | p{3cm}  |}
\hline
Step & Handling & Forventet Resultat & Test & Resultat \\
\hline
1 &  Bruger vælger at logge ud i hovedmenuen &    Mobilapplikationen informerer om krav til at logge ind med Facebook. & Aflæses på Brugers enhed & \FuckingHuge{\checkmark} \\
\hline
\end{longtable}



\chapter{Test af ikke funktionelle krav}
 \label{chp:testikkefunktionelle}
 

\subsection{Platform}

\begin{longtable}{| p{0.7cm}  | p{3cm}  | p{4cm} |  p{3cm}  | p{3cm}  |}
\hline
Step & Handling & Forventet Resultat & Test & Resultat \\
\hline
1 & Applikationen startes på en Lumia 925 & Applikationen åbner & Aflæses på Lumia 925 & \FuckingHuge{\checkmark} \\
\hline
\end{longtable}




\bibliographystyle{plainnat}
\bibliography{bibliografi}% Selects .bib file AND prints bibliography
\thispagestyle{fancy}

%\listoftodos % Make a list of todo's
\end{document}